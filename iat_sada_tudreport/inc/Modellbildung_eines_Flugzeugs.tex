\chapter{Modellbildung eines Flugzeugs}\label{cha:Modellbildung}
Die Verkopplung der beiden Flugzeuge erfordert eine gemeinsame Modellierung der Zustände der zwei Maschinen. In diesem Kapitel wird allerdings die Modellbildung von nur einem Flieger vorgestellt. Kapitel \ref{cha:ZweiFlieger} beschäftigt sich mit der Verknüpfung der dynamischen Zusammenhänge von beiden Flugzeugen, um ein Gesamtsystem zu erhalten.
\section{Zustände und zeitabhängige Variablen}
Tabelle \ref{tab:zeitVariab} dokumentiert die wichtigste zeitabhängigen Variablen, die im Rahmen des Seminars verwendet werden.
\begin{table}[h]
\centering
 \begin{tabular}{||c c c c ||} 
 \hline
 Zustand & Symbol & Einheit & Koordinatensystem \\ [0.5ex] 
 \hline\hline
 Körpergeschwindigkeit $x$-Richtung& $u$ & $\mathrm{[m/s]}$& Körperfest\\ 
 \hline
 Körpergeschwindigkeit $y$-Richtung& $v$ & $\mathrm{[m/s]}$& Körperfest\\ 
 \hline
 Körpergeschwindigkeit $z$-Richtung& $w$ & $\mathrm{[m/s]}$& Körperfest\\ 
 \hline
 Körperrotationsgeschwindigkeit  $x$-Achse& $p$ & $\mathrm{[rad/s]}$& Körperfest\\ 
 \hline
 Körperrotationsgeschwindigkeit $y$-Achse& $q$ & $\mathrm{[rad/s]}$& Körperfest\\ 
 \hline
 Körperrotationsgeschwindigkeit  $z$-Achse& $r$ & $\mathrm{[rad/s]}$& Körperfest\\ 
 \hline
 Rollwinkel & $\phi$ & $\mathrm{[rad]}$& Körperfest\\ 
 \hline
 Nickwinkel & $\theta$ & $\mathrm{[rad]}$& Körperfest\\ 
 \hline
 Gierwinkel & $\psi$ & $\mathrm{[rad]}$& Körperfest\\ 
 \hline
 Anstellwinkel & $\alpha$ & $\mathrm{[rad]}$& Körperfest\\ 
 \hline
 Schiebewinkel & $\beta$ & $\mathrm{[rad]}$& Körperfest\\ 
 \hline
 Position Nord & $x$ & $\mathrm{[m]}$& Erdkoordinaten (Flache Erde)\\ 
 \hline
 Position Ost & $y$ & $\mathrm{[m]}$& Erdkoordinaten (Flache Erde)\\ 
 \hline
 Position Unten & $z$ & $\mathrm{[m]}$& Erdkoordinaten (Flache Erde)\\ 
 \hline
 Höhenruderwinkel & $\eta$ & $\mathrm{[rad]}$& Körperfest\\ 
 \hline
 Normierte Schubkraft  & $\mathrm{sigma}_f$ & -& -\\ 
 \hline
 Querruderwinkel & $\xi$ & $\mathrm{[rad]}$& Körperfest\\ 
 \hline
 Seitenruderwinkel & $\zeta$ & $\mathrm{[rad]}$& Körperfest\\ [1ex] 
 \hline
\end{tabular}
\caption{Zeitabhängige Variablen des Modells}
\label{tab:zeitVariab}
\end{table}

Die Wahl der Zustandsvariablen muss folgende Anforderungen erfüllen, damit das System den Zweck des Projekts bewältigen kann.
\begin{itemize}
\item Zustände müssen genügen, um ein sinnvolles Modell darzustellen. 
\item Zustände müssen sinnvolle Verkopplungsbedingungen zur Verfügung stellen.
\item Das Gesamtsystem linearisiert um einen sinnvollen Arbeitspunkt (Kapitel \ref{cha:Linearisierung}) muss vollständig steuerbar und beobachtbar sein.
\item Die Linearisierung um einen konstanten Arbeitspunkt muss möglich sein.
\item Um ein zu hochdimensionales System zu vermeiden, soll die Anzahl der Zustände auf ein Minimum gehalten werden (unter der Einhaltung der vorherigen Anforderungen). 
\end{itemize}
Aus dem Grund wählt man folgende Variablen als Zustände, um ein geeignetes Flugzeugmodell zu beschreiben.
\begin{itemize}
\item Körpergeschwindigkeit $\underline{v} = [u, v, w]^T$
\item Eulerwinkel (ohne den Gierwinkel) $\underline{\varphi} = [\phi, \theta]^T$ (siehe Bild \ref{fig:KoordSyst})
\item Körperdrehgeschwindigkeit $\underline{\omega} = [p, q, r]^T$, welche die zeitliche Änderung der Eulerwinkel beschreiben.
\item Flugzeughöhe $h^\mathcal{G} = -z^\mathcal{G}$.
\end{itemize}
Der behandelte Zustandsraum für lediglich ein Flugzeug lässt sich folgendermaßen zusammenfassen\\
\begin{equation}
\underline{x}_\mathrm{f} = \begin{bmatrix} 
\underline{v} \\
\underline{\omega} \\
\underline{\varphi}\\
h^\mathcal{G}
\end{bmatrix} = \begin{bmatrix} 
u\\v\\w\\p\\q\\r\\ \phi\\ \theta\\ h
\end{bmatrix}.
\end{equation}
Der Gierwinkel $\psi$ wird nicht als Teil des Zustands beschrieben, da ansonsten $\psi$ auch als Teil des Ausgangs gewählt werden müsste, um vollständige Beobachtbarkeit zu gewährleistet, da andere Zustände von $\psi$ nicht beeinflusst werden. Dies würde in einem Verlust von Freiheitsgraden bezüglich der Wahl der Ausgänge resultieren. Variablen die ebenso nicht als Zustand berücksichtigt werden sind die Koordinaten $[x^\mathcal{G},y^\mathcal{G}]^T$. Bei der Linearisierung um einen stationären Arbeitspunkt werden die zeitlich Änderungen der Zustände zu null gesetzt. Dies würde bei $x$ und $y$ dazu führen, dass die Position des Flugzeugs konstant gehalten werden muss, was im Konflikt zu der Annahme konstanter Geschwindigkeiten $u$, $v$ und $w$ steht. Dies würde dazu führen, dass keine stabile Ruhelage gefunden werden kann, weshalb die Koordinaten nicht als Teil des Zustands berücksichtigt werden. 

\section{Stellgrößen}
Die Stellgrößen eines Flugzeugs werden als 
\begin{equation}
\underline{u}_\mathrm{f} = \begin{bmatrix} 
\eta \\ \mathrm{sigma}_f \\ \xi \\ \zeta
\end{bmatrix},
\end{equation} 
definiert, wobei die Schubkraft anhand der Gewichtskraft des Flugzeugs normiert wird: $\mathrm{sigma}_f = \dfrac{F_\mathrm{schub}}{mg}$. Bild \ref{fig:Ruder} demonstriert welche Ruder aufgeschlagen werden können, um das dynamische Verhalten des Fliegers zu beeinflussen. Das betrachtete Modell geht davon aus, dass sich das Triebwerk im Zentrum der Maschine befindet, und dass dieses zentrale Triebwerk für die gesamte Lieferung der Schubkraft verantwortlich ist.
\begin{figure}[h]
  \centering
  \begin{overpic}[width=0.5\linewidth]{./Bilder/Ruder.png}
		\put(26,37){Seitenruder}
		\put(70,25){Höhenruder}
		\put(0,15){Querruder}
		\put(42,-3){Schubkraft}
	\end{overpic}
  \caption{Bezeichnung der Rudern und Schubkraft}
  \label{fig:Ruder}
\end{figure}%\\
\subsection{Stellgrößenbeschränkung}\label{sec:StellBeschränkung}
Die Stellgrößen müssen für eine realitätsgetreue Analyse beschränkt werden. Die Autoren von \cite{RAMYoutube_Playlist} und \cite{RAMPaper} empfehlen für die Modellierung einer \textit{Boeing 757-200} folgende Einschränkung (Tabelle \ref{tab:SGF}).

\begin{table}[h]
\centering
%\begingroup
\setlength{\tabcolsep}{12pt} % Default value: 6pt
\renewcommand{\arraystretch}{1.5} % Default value: 1
 \begin{tabular}{||c c ||} 
 \hline
 Variable & Beschränkung \\ [0.5ex] 
 \hline\hline
  Höhenruderwinkel & $-25\dfrac{\pi}{180}[\mathrm{rad}]\leq\eta\leq 10\dfrac{\pi}{180}[\mathrm{rad}]$\\
  Seitenruderwinkel & $-30 \dfrac{\pi}{180}[\mathrm{rad}]\leq\zeta\leq 30\dfrac{\pi}{180}[\mathrm{rad}]$\\
  Querruderwinkel & $-25 \dfrac{\pi}{180}[\mathrm{rad}]\leq\xi\leq 25\dfrac{\pi}{180}[\mathrm{rad}]$\\
  Normierte Schubkraft & $\dfrac{\pi}{360}\leq\mathrm{sigma}_f\leq \dfrac{F_\mathrm{schub_{max}}}{mg} = \dfrac{\pi}{9}$\\ 
\hline
\end{tabular}
\caption{Stellgrößenbeschränkung}
\label{tab:SGF}
\end{table}


\section{Dynamische und kinematische Gleichungen}
\label{sec:Dynamik}
Relevant für die Herleitung der Modellgleichung auf den folgenden Seiten ist die Beziehung\\
\begin{equation}
\label{fun:Tfe}
\textbf{T}_\mathrm{fe}\underline{\dot{v}}^\mathcal{E} = \underline{\dot{v}}^\mathcal{F} + (\underline{\omega}^\mathcal{F}-\underline{\omega}^\mathcal{F}_\mathrm{e})\times\underline{v}^\mathcal{F}
\end{equation}
Der Erdrotationsgeschwindigkeitsvektor wird in der Regel als $\underline{\omega}^\mathcal{E}_\mathrm{e} = \begin{bmatrix} 
0 \\ 0 \\ 7.3\times 10^{-5} 
\end{bmatrix}\mathrm{[rad/s]}$ dargestellt, allerdings kann für erdnahe Flüge die Annahme $\underline{\omega}^\mathcal{E}_\mathrm{e} = \underline{0}$ getroffen werden.
\subsection{Dynamik}
\textbf{Translation}\\
Laut \cite{FlugmechanikBuch} ist die absolute Beschleunigung im erdfesten System  als Summe der Körperbeschleunigung  und der relativen Beschleunigung bezüglich der Erde definiert, wenn die Annahme $(\underline{\omega}_\mathrm{e}^\mathcal{E})^2 <<1$ getroffen wird
\begin{equation}
\label{fun:a}
\underline{a}^\mathcal{E} = \underline{\dot{v}}^\mathcal{E} + 2\underline{\omega}_\mathrm{e}^\mathcal{E} \times \underline{v}^\mathcal{E}.
\end{equation}
Bei der Transformation zu körperfesten Koordinaten und unter der Verwendung von Gleichung \ref{fun:Tfe}, erhält man.
\begin{align}
\textbf{T}_\mathrm{fe}\underline{a}^\mathcal{E} = \textbf{T}_\mathrm{fe}(\underline{\dot{v}}^\mathcal{E} + 2\underline{\omega}_\mathrm{e}^\mathcal{E} \times \underline{v}^\mathcal{E}),\\
\underline{a}^\mathcal{F} = \textbf{T}_\mathrm{fe}\underline{\dot{v}}^\mathcal{E} +  2\underline{\omega}_\mathrm{e}^\mathcal{F} \times \underline{v}^\mathcal{F},\\
\label{fun:a}
\Rightarrow \underline{a}^\mathcal{F} = \underline{\dot{v}}^\mathcal{F} + (\underline{\omega}^\mathcal{F}+\underline{\omega}_\mathrm{e}^\mathcal{F})\times \underline{v}^\mathcal{F}
\end{align}
Des Weiteren wird die Vereinfachung vorgenommen, dass die Erdrotationsgeschwindigkeit $\underline{\omega}_\mathrm{e}^\mathcal{F}$ vernachlässigbar ist. Die Körperbeschleunigung im körperfesten System ist, laut dem zweiten Satz von Newton, gleich der Summe aller eingreifenden Kräfte. 
\begin{equation}
\label{fun:SummeKräfte}
\underline{a}^\mathcal{F} = \dfrac{1}{m}\underline{r}^\mathcal{F}_\mathrm{g} + \textbf{T}_\mathrm{gf}\begin{bmatrix} 
0 \\ 0 \\ g 
\end{bmatrix},
\end{equation}
wobei $\underline{r}^\mathcal{F}_\mathrm{g}$ die Summe der aerodynamischen Kräfte und Schubkraft darstellt, welche in Kapitel \ref{sec:Kräfte} im Detail eingegangen wird. Die Schwerkraft lässt sich einfach im geodätischen KS darstellen. Dazu wird noch die Transformation zu körperfesten Koordinaten benötigt.\\
Mithilfe der Gleichungen \ref{fun:a} und \ref{fun:SummeKräfte} lässt sich die allgemeine translatorische Bewegungsgleichung berechnen.
\begin{equation}
\label{fund:TranslatGL}
\underline{\dot{v}} = \dfrac{1}{m}\underline{r}_\mathrm{g} + \textbf{T}_\mathrm{fg}\begin{bmatrix} 
0\\
0\\
g
\end{bmatrix} - \underline{\omega}\times\underline{v}.
\end{equation}
\textbf{Rotation}\\
Unter der Annahme, dass sich weder die Massenverteilung noch die Trägheitsmatrix $\textbf{I}$  zeitlich ändern, kann man eine vereinfachte Variante des Drallsatzes anwenden.
\begin{equation}
\label{fun:Drallsatz}
\underline{q}^\mathcal{F}_\mathrm{g} =  \underline{\omega}^\mathcal{F}\times\textbf{I}\underline{\omega}^\mathcal{F} + \textbf{I}\underline{\dot{\omega}}^\mathcal{F},
\end{equation} 
wobei $ \underline{\omega}^\mathcal{F}\times\textbf{I}\underline{\omega}^\mathcal{F}$ den \textit{Stein'scher Anteil} repräsentiert. Kapitel \ref{sec:Kräfte} beschäftigt sich mit der Herleitung der Summe des aerodynamischen und Schubmoment $\underline{q}^\mathcal{F}_\mathrm{g}$. Durch eine einfache Umformung der Gleichung \ref{fun:Drallsatz} enthält man die allgemeine rotatorische Bewegungsgleichung
\begin{equation}
\label{fund:RotGL}
\underline{\dot{\omega}} = -\textbf{I}^{-1}\underline{\omega}\times\textbf{I}\underline{\omega} + \textbf{I}^{-1}\underline{q}^\mathcal{F}_\mathrm{g}.
\end{equation} 

\subsection{Kinematik}
Die Berechnung des Zustands $h^\mathcal{G}$ erfolgt über die Differentialgleichung 
\begin{equation}
\dot{h}^\mathcal{G} = -\textbf{T}_\mathrm{l}\textbf{T}_\mathrm{gf}(\underline{\varphi}_g)\underline{v}.
\end{equation}
Der Vektor $\underline{\varphi}_\mathrm{g} = [\underline{\varphi},\psi]^T $ ermöglicht die Transformation der Geschwindigkeit von körperfesten zu geodätischen Koordinaten. Die Matrix $\textbf{T}_\mathrm{l} = [0, 0, 1]$ extrahiert nur die z-Komponente des berechneten Ortsvektors.\\
Die kinematische Differentialgleichung des Eulerwinkels ist durch die Aufintegrierung der Drehgeschwindigkeit der körperfesten bezüglich der geodätischen Koordinaten erhältlich
\begin{equation}
\label{fun:KinEuler}
\underline{\dot{\varphi}} = \textbf{T}_\mathrm{\varphi}\textbf{J}(\underline{\varphi})(\underline{\omega}-\underline{\omega_\mathrm{g}}).
\end{equation}
Da das Modell für kurze Flugdistanzen ausgelegt ist, kann man annehmen, dass die Erde während des Flugs flach ist. Daher entfällt die Drehgeschwindigkeit des geodätischen Systems $\underline{\omega_\mathrm{g}}$. Die Matrix $\textbf{T}_\mathrm{\varphi}$ liest nur $\underline{\varphi}$ aus dem gesamten Eulerwinkelvektor $\underline{\varphi_\mathrm{g}}$. Letztlich ist die Matrix $\textbf{J}(\underline{\varphi})$ verantwortlich für die Drehung der Koordinaten, die die Eulerwinkel auf das körperfeste Bezugssystem umwandeln.
\begin{equation}
\textbf{J}(\underline{\varphi}) = \dfrac{1}{\cos(\theta)}\begin{bmatrix} 
\cos(\theta) & \sin(\theta)\sin(\phi)&\cos(\phi)\sin(\theta)\\
0&\cos(\phi)\cos(\theta)&-\sin(\phi)\cos(\theta)\\
0&\sin(\phi)& \cos(\phi)
\end{bmatrix}
\end{equation}
\subsection{Blockschaltbild der Bewegungsgleichung}
Bild \ref{fig:BSB} zeigt die vorgestellten dynamischen Zusammenhänge mithilfe eines Blockschaltbildes.
\begin{figure}[h]
  \centering
  \begin{overpic}[width=1.05\linewidth]{./Bilder/BSBDynKin.png}
		\put(8,17.5){$\dfrac{1}{m}$}
		\put(16,8){$\textbf{T}_\mathrm{fg}$}
		\put(37.5,17.5){$\dfrac{1}{s}$}
		\put(67,17.5){$-\textbf{T}_\mathrm{l}$}
		\put(79,17.5){$\textbf{T}_\mathrm{gf}$}
		\put(90,17.5){$\dfrac{1}{s}$}
	     \put(0,20){$\underline{r}^\mathcal{F}_\mathrm{g}$}
	     \put(0,44){$\underline{q}^\mathcal{F}_\mathrm{g}$}
	     \put(28,41.5){$\textbf{I}^{-1}$}
	     \put(47,41.5){$\dfrac{1}{s}$}
	     \put(67,41.5){$\textbf{T}_\mathrm{\varphi}$}
	     \put(79,41.5){$\textbf{J}(\underline{\varphi})$}
	     \put(90,41.5){$\dfrac{1}{s}$}
	     \put(35,25.5){$\underline{\omega}\times\underline{v}$}
	     \put(34,35.5){$\underline{\omega}\times\textbf{I}\underline{\omega}$}
	     \put(19,20){-}
	     \put(16.5,40){-}
	     \put(54,47){Dynamik}
	     \put(63,47){Kinematik}
	      \put(18,1){$[0,0,g]^T$}
	      \put(58,18.5){$\underline{v}$}
	      \put(58,43){$\underline{\omega}$}
	       \put(94,43.5){$\underline{\varphi}$}
	       \put(94,18.5){$h^\mathcal{G}$}
	\end{overpic}
	\caption{Blockschaltbild der Dynamik und Kinematik}
	\label{fig:BSB}
\end{figure}%\\
\section{Anströmungsgrößen und Atmosphäre}
\label{sec:atmospäre}
Nach der Ermittlung der dynamischen Zusammenhänge ist es an dieser Stelle möglich die Anströmungsgrö{\ss}en darzustellen, welche in den folgenden Kapiteln benötigt werden. Die Abbildung \ref{fig:KoordSyst} zeigt, dass der Anstellwinkel $\alpha$ und der Schiebewinkel $\beta$ die Drehung des aerodynamischen Bezugssystems bezüglich der körperfesten Koordinaten repräsentieren. Die gesamte Anströmungsgeschwindigkeit $\underline{v_\mathrm{a}}$ ist von der Bahngeschwindigkeit des Körpers und der Windgeschwindigkeit abhängig. Au{\ss}erdem bestimmt $\underline{v_\mathrm{a}}$ die $x$-Richtung der aerodynamischen Koordinaten. Die Berechnung dieser Grö{\ss}en und der atmosphärischen Eigenschaften werden folgenderma{\ss}en festgelegt
\begin{itemize}
\item Anströmungsgeschwindigkeitsvektor $\underline{v}_\mathrm{A} = \underline{v}- \underline{v}_\mathrm{w}  = [u_\mathrm{A},v_\mathrm{A},w_\mathrm{A}]^T $, wobei $\underline{v}_\mathrm{w}$ den Windgeschwindigkeitsvektor darstellt. Die Windgeschwindigkeit wird im Rahmen des Projekts als externe Störung (Kapitel \ref{cha:ZweiFlieger}) betrachtet  und wird daher nicht weiter als Teil des Modells berücksichtigt.  
\item Betrag der Anströmungsgeschwindigkeit $V_\mathrm{A} = |\underline{v}_\mathrm{A}| $.
\item Anstellwinkel $\alpha = \tan^{-1}(\dfrac{w_\mathrm{A}}{u_\mathrm{A}})$
\item Schiebewinkel $\beta = \sin^{-1}(\dfrac{v_\mathrm{A}}{V_\mathrm{A}})$
\item Schwerkraft $g = 9.81 \mathrm{m/s^2} = konst.$
\item Lufttemperatur $ T = 288.15 \mathrm{K} - 0.0065h^\mathcal{G}\mathrm{K/m}$ (siehe Autor \cite{AircraftCS})
\item Statischer Luftdruck $p_\mathrm{s} = 101325\mathrm{N/m^2}(\dfrac{T}{288.15  \mathrm{K}})^{\dfrac{g}{1.86584 \mathrm{m/s^2}}}$
\item Luftdichte $\rho = \dfrac{p_\mathrm{s}}{287.053T/\mathrm{K}} \mathrm{Kg/Nm}$
\item Staudruck $q_\mathrm{s} = \dfrac{1}{2}\rho V_\mathrm{A}^2$

\end{itemize}
\section{Geometrische Beiwerte}
\label{sec:Beiwerte}
Die aerodynamischen Beiwerte sind dimensionslose Grö{\ss}en, die dafür verantwortlich sind die aerodynamische Kräfte und Momente zu bestimmen. Sie hängen von der geometrischen Anordnung des Fliegers (bspw. Ruderwinkel), den Anströmungsgrö{\ss}en und dem aerodynamischen Verhalten eines Körpers in einem Windtunnel ab. Die in diesem Kapitel vorgestellten Koeffizienten spiegeln die aerodynamische Eigenschaft des \textit{Boeing 757-200} wider. Für weitere Details dieser Konstanten sei auf \cite{AircraftCS}, \cite{RAMPaper} und \cite{RAMYoutube_Playlist} verwiesen. Die Gleichungen zur Bestimmung der Koeffizienten sind von den Werken von \cite{FlugmechanikBuch} und \cite{FM1Skrypt} abgeleitet worden. 

\subsection{Translatorische Koeffizienten}
Die aerodynamischen Kräfte werden sinnvollerweise im aerodynamischen Koordinatensystem definiert. Daher werden die folgenden Koeffizienten auch in diesem Bezugssystem vorgestellt. Die relevanten Beiwerte beeinflussen den Auftrieb, den Luftwiderstand und die Querkraft. Die Werte der aerodynamischen Konstanten sind mithilfe der Tabelle \ref{tab:KonstBeiw} definiert. 

\begin{table}[h]
\centering
 \begin{tabular}{||c c c  ||} 
 \hline
 Variable Bezeichnung & Wert & Einheit\\ 
 \hline\hline
 $K_\alpha$ & $5.5$ & $\mathrm{[rad]^{-1}}$ \\ 
 \hline
 $\alpha_0$ & $-11.5\dfrac{\pi}{180}$ & $\mathrm{[rad]}$ \\ 
 \hline
 $K_{\epsilon}$ & $0.25$ & - \\ 
 \hline
 $\alpha_\mathrm{max}$ & $14.5\dfrac{\pi}{180}$ & $\mathrm{[rad]}$ \\ 
 \hline
 $C_\mathrm{W_0}$ & $0.13$ & - \\ 
 \hline
 $\kappa$ & $0.07$ & -\\ 
 \hline
 $C_\mathrm{A_0}$ & $0.45$ & -\\ 
 \hline
 $K_\beta$ & $-1.6$ & -\\  
 \hline
 $K_\zeta$ & $0.24$ & -\\ 
 \hline
\end{tabular}
\caption{Aerodynamischen Koeffizienten eines \textit{Boeing 757-200}}
\label{tab:KonstBeiw}
\end{table}
\textbf{Auftriebskoeffizient}\\
Abbildung \ref{fig:CA} zeigt den generellen Verlauf zur Bestimmung des Flügelauftriebskoeffizienten $C_\mathrm{A_F}^\mathcal{A}$. Wenn $-\alpha_\mathrm{max}\leq|\alpha|\leq\alpha_\mathrm{max}$ verläuft die Kurve näherungsweise linear. Über diesen Punkt hinaus entsteht eine kubische Beziehung. Um die Linearisierung (Kapitel \ref{cha:Linearisierung}) zu vereinfachen, wird angenommen, dass die Funktion ein globales lineares Verhalten aufweist.
\begin{equation}
C_\mathrm{A_F}^\mathcal{A} = K_\alpha(\alpha - \alpha_0).
\end{equation} 
Dies verursacht, dass für $|\alpha|> \alpha_\mathrm{max}$ das System etwas von der Realität abweicht. Die linearen Winkelbeziehungen führen dazu, dass eine Drehung über $90^{\circ}$  ausgeschlossen werden muss. Diese Annahme sollte für Passagierflugzeuge selbstverständlich erhalten werden.\\
Das Höhenruder verursacht ebenfalls eine Auftriebskraft. Dabei muss man den Abwind $\epsilon$ berücksichtigen, der von der Flügeldurchströmung verursacht wird
\begin{equation}
\epsilon = K_{\epsilon}(\alpha-\alpha_\mathrm{0}).
\end{equation}
Damit wird es möglich den Anstellwinkel $\alpha_\mathrm{h}$ an dem Höhenruder in Abhängigkeit des Anstellwinkels, Höhenruderwinkels und Anströmungsgeschwindigkeit auszurechnen. Dabei ist der Auftriebskoeffizient am Höhenruder auch berechenbar.
\begin{align}
\alpha_\mathrm{h} = \alpha - \epsilon-\eta + q\dfrac{l_\mathrm{t}}{V_\mathrm{A}}\\
\Rightarrow C_\mathrm{A_H}^\mathcal{A} = 3.1 \alpha_\mathrm{h}\dfrac{S_\mathrm{H}}{S}.
\end{align}
Letztlich lässt sich der gesamte Auftriebskoeffizient aus der Summe der beiden ermitteln.
\begin{equation}
C_\mathrm{A}^\mathcal{A} = C_\mathrm{A_F}^\mathcal{A}+C_\mathrm{A_H}^\mathcal{A}
\end{equation}

\textbf{Widerstandskoeffizient}\\
Bild \ref{fig:CW} veranschaulicht die Beziehung zwischen $C_\mathrm{W}^\mathcal{A}$ und $C_\mathrm{A_F}^\mathcal{A}$. Diese lässt sich durch folgende Funktion abbilden
\begin{equation}
C_\mathrm{W}^\mathcal{A} = C_\mathrm{W_0} + \kappa(C_\mathrm{A_F}^\mathcal{A}-C_\mathrm{A_0})^2.
\end{equation}
$C_\mathrm{W}^\mathcal{A}$ scheint an der ersten Stelle von dem Schiebewinkel unabhängig zu sein. Allerdings ist durch die Transformation zu körperfesten Koordinaten  die Abhängigkeit dieses Winkels gegeben. Dies wird dazu führen, dass eine Vergrö{\ss}erung von $\beta$ ($|\beta| \leq 90^{\circ}$) eine höhere Widerstandskraft verursachen würde.\\
\textbf{Querkraftkoeffizient}\\
Letztlich hängt der Querkraftkoeffizient linear von dem Schiebewinkel und dem Seitenruderwinkel ab.
\begin{equation}
C_\mathrm{Q}^\mathcal{A} = K_\beta\beta + K_\zeta\zeta
\end{equation}
\begin{figure}[h]

%\centering
\begin{subfigure}{0.49\textwidth}
  \centering
  \begin{overpic}[width=1\linewidth]{./Bilder/BeiwertAlpha.png}
		\put(7,68){$C_\mathrm{A_F}$}
		\put(50,35){$K_\alpha$}
		\put(90,0){$\alpha \mathrm{[rad]}$}
		\put(68,0){$\alpha_\mathrm{max}$}
		\put(3,0){$\alpha_0$}
		
	
	\end{overpic}
  \caption{Auftriebskoeffizient}
  
  \label{fig:CA}
\end{subfigure}%
\begin{subfigure}{0.49\textwidth}
  \centering
   \begin{overpic}[width=1\linewidth]{./Bilder/BeiwertCW.png}
		\put(7,68){$C_\mathrm{A_F}$}
		\put(90,0){$C_\mathrm{W}$}
		\put(7,20){$C_\mathrm{A_0}$}
		\put(26,-2){$C_\mathrm{W_0}$}
		
	
	\end{overpic}
  \caption{Auftriebskoeffizient und Widerstandskoeffizient }
  %\label{fig:sub2}
\end{subfigure}
\caption{Darstellung der aerodynamischen Koeffizienten \cite{FlugmechanikBuch}}
\label{fig:CW}
\end{figure}
\subsection{Rotatorische Koeffizienten}
Laut \cite{RAMYoutube_Playlist} und \cite{RAMPaper} erfolgt die Berechnung der Momente in körperfesten Koordinaten. Die Roll-, Nick- und Gierbeiwerte lassen sich durch folgende Gleichung ermitteln
\begin{equation}
\underline{c}_\mathrm{m}  = \begin{bmatrix} 
C_\mathrm{l}\\
C_\mathrm{m}\\
C_\mathrm{n}
\end{bmatrix}=\underline{n} + \dfrac{\partial  \underline{c}_\mathrm{m}}{\partial  \underline{c}_x} \underline{\omega} + \dfrac{\partial  \underline{c}_\mathrm{m}}{\partial  \underline{c}_u} \tilde{\underline{u}}.
\end{equation}
Der Vektor $\underline{n}$ bildet den statischen Effekt (ohne Berücksichtigung der Auslenkung und Drehgeschwindigkeit) der aerodynamischen Winkel auf die Momente ab. 
\begin{equation}
\underline{n} = \begin{bmatrix} 
-1.4\beta \\
-0.59 -3.1(\alpha-\epsilon)\dfrac{S_\mathrm{H}l_\mathrm{H}}{Sc} \\
(1 - \alpha\dfrac{180}{15\pi})\beta
\end{bmatrix}.
\end{equation}
Der negative Gradient von $\alpha$ in der zweiten Zeile von $\underline{n}$ stellt sicher, dass die statische Längsstabilität gewährleistet ist. Dies hei{\ss}t, dass bei kleiner Abweichung des statischen Flugverhaltens der Nickwinkel wieder abklingen kann.\\
Die Jakobimatrix des dynamischen Drehverhaltens $\dfrac{\partial  \underline{c}_\mathrm{m}}{\partial  \underline{c}_x}$ stellt die Abhängigkeit der Momenten bezüglich der Drehgeschwindigkeiten und der Anstömungsgeschwindigkeit dar.
\begin{equation}
\dfrac{\partial  \underline{c}_\mathrm{m}}{\partial  \underline{c}_x} = \dfrac{c}{V_\mathrm{A}}\begin{bmatrix} 
-11 & 0 & 5 \\
0 &-4.03\dfrac{S_\mathrm{H}l_\mathrm{H}^2}{Sc}&0 \\
1.7 & 0 & -11.5
\end{bmatrix}
\end{equation}
Mithilfe der Jakobimatrix des Steuerverhaltens $\dfrac{\partial  \underline{c}_\mathrm{m}}{\partial  \underline{c}_u}$ kann die Ruderauslenkung $\tilde{\underline{u}} = [\xi,\eta,
\zeta]^T$ das Moment beeinflussen. Dies spielt bei der Sicherstellung der Steuerbarkeit der Strecke eine essentielle Rolle. 
\begin{equation}
\dfrac{\partial  \underline{c}_\mathrm{m}}{\partial  \underline{c}_u} = \begin{bmatrix} 
-0.6 & 0 & 0.22 \\
0 &-3.1\dfrac{S_\mathrm{H}l_\mathrm{H}}{Sc}&0 \\
0 & 0 & -0.63
\end{bmatrix}.
\end{equation}
Die meisten Einträge von $\dfrac{\partial  \underline{c}_\mathrm{m}}{\partial  \underline{c}_u}$ und  $\dfrac{\partial  \underline{c}_\mathrm{m}}{\partial  \underline{c}_x}$ sind von den Dimensionen des Flugzeugs unabhängig. Dies ist eine Vereinfachung, die von \cite{RAMPaper} vorgeschlagen wird, um die Modellkomplexität deutlich  zu verringern. Au{\ss}erdem sind die Beziehungen der aerodynamischen Winkel linear, was wieder dazu führt, dass Drehungen über $90^{\circ}$ nicht erlaubt sind, um die physikalischen Grenzen einzuhalten.
\section{Kräfte und Momente}
\label{sec:Kräfte}
Nachdem die geometrischen Beiwerte ermittelt worden sind, ist die Berechnung aller Kräfte und Momente möglich.
\subsection{Triebwerk}
Die resultierende Schubkraft wird mit $\mathrm{sigma}_f$ gesteuert
\begin{equation}
F_\mathrm{res} = \mathrm{sigma}_f mg.
\end{equation}
Daher ist der gesamte Schubkraftsvektor folgenderma{\ss}en festgelegt
\begin{equation}
\underline{r}^\mathcal{F}_\mathrm{schub} = F_\mathrm{res}\begin{bmatrix} 
\cos(i_\mathrm{f})\\
0\\
\sin(i_\mathrm{f})
\end{bmatrix} = F_\mathrm{res}\begin{bmatrix} 
1\\
0\\
0
\end{bmatrix}.
\end{equation}
Da man annimmt, dass der Drall des Propellers vernachlässigbar ist, kann das Schubmoment einfach aus der Schubkraft und dem Abstand zwischen dem Schwerpunkt und der Lage des Triebwerks ermittelt werden.
\begin{equation}
\underline{q}^\mathcal{F}_\mathrm{schub} = (\underline{p}_\mathrm{schub}-\underline{p}_\mathrm{sp})\times\underline{r}^\mathcal{F}_\mathrm{schub}
\end{equation} 
\subsection{Aerodynamik}
Die aerodynamischen Zusammenhänge von Kräften und Momenten in Abhängigkeit der geometrischen Beiwerten sind wie folgt festgelegt.
\begin{itemize}
\item Auftriebskraft: $A = q_\mathrm{s} SC_\mathrm{A}$
\item Widerstandskraft: $W = q_\mathrm{s} SC_\mathrm{W}$
\item Querkraft: $Q = q_\mathrm{s} SC_\mathrm{Q}$
\item Rollmoment: $L_A = q_\mathrm{s} SbC_\mathrm{l}$
\item Nickmoment: $M_A = q_\mathrm{s} ScC_\mathrm{m}$
\item Giermoment: $N_A = q_\mathrm{s} SbC_\mathrm{n}$
\end{itemize}
Das Freikörperbild \ref{fig:Freikörperbild} veranschaulicht wie sich die Kräfte im aerodynamischen Bezugssystem auswirken. Mithilfe des Bildes ist die Richtung des aerodynamischen Kraftvektors sofort ersichtlich. 
\begin{equation}
\underline{r}^\mathcal{A}_\mathrm{a} = \begin{bmatrix} 
-W\\
Q\\
-A
\end{bmatrix}.
\end{equation}

\begin{figure}[h]
  \centering
  \begin{overpic}[width=\linewidth]{./Bilder/Freikoerperbild.png}
		\put(0,20){$x_\mathrm{a}$}
		\put(20,12){$z_\mathrm{a}$}
		\put(53,23){$x_\mathrm{a}$}
		\put(74,28){$y_\mathrm{a}$}
		\put(25,30){$A$}
		\put(33,20){$W$}
		\put(77,31){$Q$}
		\put(85,18){$W$}
	
	\end{overpic}
	\caption{Freikörperbild der aerodynamischen Kräften}
	\label{fig:Freikörperbild}
\end{figure}

Die Kraft $\underline{r}^\mathcal{A}_\mathrm{a}$ muss allerdings in das körperfeste Koordinatensystem umgewandelt werden.
\begin{equation}
\underline{r}^\mathcal{F}_\mathrm{a} = \textbf{T}_\mathrm{fa}(\alpha,\beta)\underline{r}^\mathcal{A}_\mathrm{a}
\end{equation}
Das gesamte aerodynamische Moment lässt sich mithilfe des Roll-, Nick- und Giermomentes, der aerodynamischen Kraft und der Verschiebung des aerodynamischen Schwerpunktes vom Massenschwerpunkt ausrechnen
\begin{equation}
\underline{q}^\mathcal{F}_\mathrm{a} = \begin{bmatrix} 
L_\mathrm{A}\\
M_\mathrm{A}\\
N_\mathrm{A}
\end{bmatrix} + (\underline{p}_\mathrm{ac}^\mathcal{F}-\underline{p}_\mathrm{sp}^\mathcal{F})\times\underline{r}^\mathcal{F}_\mathrm{a} = \begin{bmatrix} 
L_\mathrm{A}\\
M_\mathrm{A}\\
N_\mathrm{A}
\end{bmatrix}
\end{equation}
Schlie{\ss}lich kann man die Gesamtkraft und das Gesamtmoment für die Dynamik des Modells (Kapitel \ref{sec:Dynamik} ) zur Verfügung stellen.
\begin{align}
\underline{r}^\mathcal{F}_\mathrm{g} = \underline{r}^\mathcal{F}_\mathrm{a} + \underline{r}^\mathcal{F}_\mathrm{schub}\\
\underline{q}^\mathcal{F}_\mathrm{g} = \underline{q}^\mathcal{F}_\mathrm{a} + \underline{q}^\mathcal{F}_\mathrm{schub}
\end{align}

\section{Fehlende Größen zur ausführlichen Visualisierung des Modells}
\label{sec:Visialisis}
Für die vollständige Visualisierung des Modells, werden noch die Positionen $[x^\mathcal{E},y^\mathcal{E},z^\mathcal{E}]^T$ und der Gierwinkel $\psi$ benötigt. Die erdfesten Koordinaten (Index $\mathcal{E}$) unterscheiden sich von den geodätischen nur am Mittelpunkt, der in den erdfesten Koordinaten auf der Erdoberfläche liegt. Die Transformation erwartet den Bahnazimutwinkel $\lambda$ und den Bahnneigungswinkel $\gamma$.
\begin{align}
\underline{v}^\mathcal{G} = \textbf{T}(\underline{\varphi})_\mathrm{gf}\underline{v} = [u^\mathcal{G},v^\mathcal{G},w^\mathcal{G}]^T\\
V_\mathrm{k} = |\underline{v}^\mathcal{G}|\\
\gamma = -\sin^{-1}(\dfrac{w^\mathcal{G}}{V_\mathrm{k}})\\
\lambda = \tan^{-1}(\dfrac{v^\mathcal{G}}{u^\mathcal{G}})\\
\dot{x}^\mathcal{E} = V_\mathrm{k}\cos(\gamma)\cos(\lambda)\\
\dot{y}^\mathcal{E} = V_\mathrm{k}\cos(\gamma)\sin(\lambda)\\
\dot{z}^\mathcal{E} = -V_\mathrm{k}\sin(\gamma)\\
\end{align}
Die DGL für $\psi$ wird ermittelt, indem man zunächst den Gierwinkel als Teil des Zustands betrachtet. Daher kann die Differentialgleichung ausgelesen werden. In Folge entfernt man wieder den Gierwinkel aus dem gesamten Zustand. Dies ist unproblematisch, da keine andere Variable von $\psi$ abhängt. Seine DGL lautet:
\begin{equation}
\dot{\psi} = r(\cos(\phi)\cos(\theta)) + q(\sin(\phi)\sin(\theta)).
\end{equation}

\section{Fazit der Modellannahmen}
Die Hauptannahmen, die während der Modellbildung getroffen worden sind, werden folgenderma{\ss}en zusammengefasst.
\begin{itemize}
\item Die betrachtete Erde sei flach: $\omega_\mathrm{g} = 0$.
\item Erdnaher Weltraum: $\omega_\mathrm{e} = 0$.
\item Konstante Schwerkraft.
\item Konstante Masse. Allerdings wird die Masse sich während der Betankung verändern (Kapitel \ref{cha:ZweiFlieger}).
\item Keine Änderung der Massenverteilung am Flugzeug: $\dot{\textbf{I}} = 0$.
\item Starres Flugzeug ohne bewegte Teile.
\item Drall der Turbine sei vernachlässigbar.
\item Turbine sei im Zentrum des Flugzeugs zusammengefasst und die Modellierung der Schubkraft sei vereinfacht.
\item Direkte Vorgabe der Stellgrö{\ss}en ohne Zeitverzögerung: Dynamik der Rudermotoren sei Optimal.
\item Aerodynamischer Schwerpunkt sei gleich dem Massenschwerpunkt: $P_\mathrm{sp} = P_\mathrm{ac}$.
\item Keine kubische Beziehung von $C_\mathrm{A_F}(\alpha)$: Abweichung von Sollwerten bei $|\alpha|\geq 14.5^{\circ}$.
\item Die meisten Beziehungen von Winkeln und Beiwerten seien linear: Das Modell verliert die komplette physikalische Gültigkeit bei Drehungen über $90^{\circ}$, was für Passagierflugzeug schon vorausgesetzt wird.

\end{itemize}

\section{Eingliederung und Zusammenfassung des Modells}
Bild \ref{fig:Gliederung} stellt alle Zusammenhänge der Bestandteile der Modellbildung, sowie den Bezug auf die einzelnen Kapitel dar.
\begin{figure}[h]
  \centering
  \begin{overpic}[width=0.9\linewidth]{./Bilder/ModellBSB.png}
        %Title
        \put(3,95){Visualisisierung}
        \put(3,66){Strecke}
        %Blöcke
		\put(12,52){Beiwerte}
		\put(11,48){Kapitel \ref{sec:Beiwerte}}
		\put(41,52){Kräfte/Momente}
		\put(45,48){Kapitel \ref{sec:Kräfte}}
		\put(75,52){Dynamik}
	    \put(75,49){Kinematik}
		\put(74.5,46){Kapitel \ref{sec:Dynamik}}
		\put(73,86){Visualisierung}
		\put(75,82){Kapitel \ref{sec:Visialisis}}
		\put(39,92){Aerospace Toolbox}
		\put(47,89){\cite{MatlabBild}}
	    \put(74,21){Anströmung}
	    \put(74,18){Atmosphäre}
		\put(75,15){Kapitel \ref{sec:atmospäre}}
		%Signale
		\put(1,48){$\underline{u}$}
	
		\put(27.5,53){$C_\mathrm{A},C_\mathrm{W},C_\mathrm{Q}$}
		\put(27.5,43){$C_\mathrm{l},C_\mathrm{m},C_\mathrm{n}$}
		\put(65,60){$\underline{r}_\mathrm{g}^\mathcal{F}$}
		\put(65,43){$\underline{q}_\mathrm{g}^\mathcal{F}$}
		
		\put(95,60){$\underline{v}$}
		\put(95,55){$\underline{\omega}$}
		\put(95,49){$\underline{\varphi}$}
		\put(95,43){$h^\mathcal{G}$}
		
		\put(74,65){$\underline{v}$}
		\put(83,65){$\underline{\omega}$}
		\put(90,65){$\underline{\varphi}$}
		
		\put(74,35){$\underline{v}$}
		\put(83,35){$\underline{\omega}$}
		\put(90,35){$\underline{\varphi}$}
		
		\put(65,27){$q_\mathrm{s}$}
		\put(65,21){$\alpha$}
		\put(65,16){$\beta$}
		\put(65,12){$V_\mathrm{A}$}
		\put(65,9){$\underline{\omega}$}
		
		\put(60.25,95){$[x^\mathcal{E},y^\mathcal{E},z^\mathcal{E}]^T$}
		\put(62,77){$[\underline{\varphi},\psi]^T$}
		
	\end{overpic}
	\caption{Eingliederung des Modells}
	\label{fig:Gliederung}
\end{figure}

