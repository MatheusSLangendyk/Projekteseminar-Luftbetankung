\chapter{Notation und Parameter}\label{cha:Notation}

Dieses Kapitel beschäftigt sich mit der Festlegung aller Notationen, die im Laufe des Berichtes angewandt werden. Außerdem werden anschließend alle Koordinatensystemen vorgestellt, welche die Modellbildung ermöglichen.

\section{Notationen}
\begin{itemize}
\item \textbf{Matrizen}: Als Matrizen werden Elemente von $\mathbb{R}^{a\times b}$ bezeichnet, $\forall a,b \in \mathbb{N}$ und $b>1$. Die Variablen werden groß und fett geschrieben, zum Beispiel $\textbf{M} = \begin{bmatrix} 
1 & 0\\
0 & 1
\end{bmatrix}$
\item \textbf{Vektoren}: Vektoren sind Elemente von $\mathbb{R}^{a\times 1}$ (stehender Vektor), wobei $a \in \mathbb{N}$ und $a>1$. Die Variablen werden klein und unterstrichen gekennzeichnet. Beispiel dafür wäre: $\underline{v} = [1,0]^T = \begin{bmatrix} 
1 \\
0 
\end{bmatrix}$
\item \textbf{Skalare}: Skalare können sowohl klein als auch groß geschrieben werden. Zum Beispiel $S_\mathrm{a} = 1$ oder auch $s_\mathrm{b} = 2$
\item \textbf{Koordinaten}: Im Rahmen dieses Projektseminars werden unterschiedliche Koordinatenbezugssysteme verwendet. Die Unterscheidung erfolgt über einen kaligraphischen Index. Der Vektor $\underline{v}_\mathrm{a}^\mathcal{F}$ würde zum Beispiel dem körperfesten Koordinatensystem $\mathcal{F}$ gehören. Die Beschreibung aller relevanten Bezugssysteme wird am Ende dieses Kapitel vorgestellt.
\item \textbf{Transformationsmatrizen}: Die Transformationsmatrizen, die die Koordinatensysteme transformieren, werden beispielsweise als $\textbf{T}_\mathrm{xy}$  dargestellt. Dabei transformiert wandelt man das Koordinatensystem y in das Koordinatensystem x um.
\end{itemize}
\section{Zeitlich konstante Parameter}
Die Modellierung des Flugzeugs bezieht sich auf die Eigenschaften eines \textit{Boeing 757-200}. Die im Modell verwendeten konstanten Parameter sind von \cite{B7571, B7572,RAMPaper} hergeleitet worden. Tabelle \ref{tab:Parameter} stellt die zeitlich invarianten Konstanten bezüglich der Geometrie der Maschine dar. Die Schwerkraft wird ebenso als konstant mit dem Wert $g = 9,81 \mathrm{m/s}$ angenommen.

\begin{table}[h]
 \begin{tabular}{||c c c c ||} 
 
 \hline
 Parameter & Symbol & Einheit & Wert \\ [0.5ex] 
 \hline\hline
 Masse & $m$& $\mathrm{[kg]}$& 120000\\ 
 \hline
 Flügelfläche & $S$& $\mathrm{[m^2]}$& 260\\ 
 \hline
 Fläche des Höhenruders  & $S_\mathrm{t}$& $\mathrm{[m^2]}$& 54\\ 
 \hline
 Generalisierte Länge & $l$& $\mathrm{[m]}$&6.6\\ 
 \hline
 Abstand SP und aerodyn. SP des Höhenruders & $l_\mathrm{t}$ & $\mathrm{[m]}$& 24.8\\ 
 \hline
 Spannweite & $b$& $\mathrm{[m]}$& 44.8\\ 
 \hline
Flügeltiefe & $c$& $\mathrm{[m]}$&6.6\\ 
 \hline
Trägheitstensor & $\textbf{I}$& $\mathrm{[kg m^2]}$&$\dfrac{m}{ \mathrm{[kg]}}\begin{bmatrix} 
40.7 & 0& 2.09\\
0 & 64 & 0\\
2.09& 0& 99.2
\end{bmatrix}$\\ 
 \hline
 Position des Massenschwerpunktes & $\underline{p}_\mathrm{sp}^\mathcal{F}$& $\mathrm{[m]}$& $[0,0,0]^T$\\ 
  \hline
 Position des aerodynamischen Schwerpunktes & $\underline{p}_\mathrm{ac}^\mathcal{F}$& $\mathrm{[m]}$& $[0,0,0]^T$\\
  \hline
  Position des Triebwerks bzgl. des Schwerpunktes & $\underline{p}_\mathrm{schub}^\mathcal{F}$& $\mathrm{[m]}$& $[0,0,1.9]^T$ \\ 
  \hline
Einbauwinkel des Triebwerks bzgl. Längsachse & $i_\mathrm{f}$& $\mathrm{[rad]}$& 0\\ 
 \hline
 Maximale Schubkraft &$F_\mathrm{max}$ & $\mathrm{[N]}$& 412020\\  [1ex] 
 \hline
\end{tabular}
\caption{Zeitlich-invariante Geometrieparameter des \textit{Boeing 757-200}. SP steht für Schwerpunkt.}
\label{tab:Parameter}
\end{table}
Bild \ref{fig:Parameter}  veranschaulicht die Konstanten bezüglich der Geometrie eines Flugzeugs.

\begin{figure}[h]
%\centering
\begin{subfigure}{0.49\textwidth}
  \centering
  \begin{overpic}[width=0.8\linewidth]{./Bilder/ParameterGeomFlieger.png}
		\put(35,75){$\dfrac{S}{2}$}
		\put(35,20){$\dfrac{S}{2}$}
		\put(75,60){$\dfrac{S_\mathrm{t}}{2}$}
		\put(75,30){$\dfrac{S_\mathrm{t}}{2}$}
		\put(38,45){$\underline{p}_\mathrm{sp}$}
		\put(70,50){$l_\mathrm{t}$}
		\put(65,70){$b$}
	
	\end{overpic}
  \caption{Flugzeug von oben}
  %\label{fig:sub1}
\end{subfigure}%
\begin{subfigure}{0.49\textwidth}
  \centering
   \begin{overpic}[width=0.8\linewidth]{./Bilder/FluegelSegment.png}
		\put(35,33){$c$}
		\put(35,13){$l$}
		
	
	\end{overpic}
  \caption{Aerodynamisches Segment eines Flügels}
  %\label{fig:sub2}
\end{subfigure}
\caption{Darstellung der geometrischen Parametern}
\label{fig:Parameter}
\end{figure}
 
%
\section{Koordinatensysteme}
Im Rahmen dieses Projekts werden verschiedene Koordinatensysteme (KS) angewandt, mit dem Ziel einige Zusammenhänge bei der Modellbildung zu vereinfachen. Dabei wählt man das geeignete Koordinatensystem für unterschiedliche Anwendungen. Tabelle \ref{tab:KoordSystem} fasst die relevanten KS und deren Anwendungen zusammen, die von den Autoren \cite{FlugmechanikBuch} vorgeschlagen worden sind. Die Zeile \textit{Anwendung} erläutert welche Kräfte und Zustände bei dem gegebenen Bezugssystem am einfachste beschrieben werden.
\begin{table}[h]
 \begin{tabular}{||c c c c ||} 
 \hline
 &Geodätisches KS  & Aerodynamisches KS & Körperfestes KS\\ [0.5ex] 
\hline
Index&$\mathcal{G}$  & $\mathcal{A}$ & $\mathcal{F}$\\
\hline
Anwendung&  \makecell{Schwerkraft und \\Lage des Fliegers} &  \makecell{Aerodynamische\\ Kräfte} &  \makecell{Schubkraft und \\Geschwindigkeiten} \\
\hline
Mittelpunkt& \makecell{Massenschwerpunkt\\ des Körpers} &  \makecell{Aerodynamischer \\Schwerpunkt} &  \makecell{Massenschwerpunkt\\ des Körpers}\\
\hline
Lage &$[x_\mathrm{g},y_\mathrm{g},z_\mathrm{g}]^T$  & $[x_\mathrm{a},y_\mathrm{a},z_\mathrm{a}]^T$ &$[x_\mathrm{f},y_\mathrm{f},z_\mathrm{f}]^T$\\
\hline
x-Richtung& Nach Norden    & Anströmungsgeschwindigkeit &  \makecell{Richtung der\\ Flugzeuglängsachse}\\
\hline
y-Richtung& Nach Osten  &  \makecell{Senkrecht auf $x_\mathrm{a}$\\ nach rechts}  &  \makecell{Senkrecht auf $x_\mathrm{f}$\\ nach rechts}\\
\hline
z-Richtung & \makecell{Senkrecht auf $x_\mathrm{g}$\\ zum Erdmittelpunkt}& \makecell{Senkrecht auf $x_\mathrm{a}$ \\in Symmetrieebene nach unten}&  \makecell{Senkrecht auf $x_\mathrm{f}$\\in Symmetrieebene nach unten}\\ [1ex] 
 \hline
\end{tabular}
\caption{Darstellung von den angewandten Koordinatensystemen (KS).}
\label{tab:KoordSystem}
\end{table}
Im Weiteren wird angenommen, dass das körperfeste KS das \textit{Hauptbezugssystem} repräsentiert. Zur Vereinfachung werden daher die Variablen für dieses KS nicht immer mit dem Index $\mathcal{F}$ gekennzeichnet. Bsp: $\underline{v} = \underline{v}^\mathcal{F}.$ Bild \ref{fig:KoordSyst} veranschaulicht die drei genannte Koordinatensystem anhand der Richtung eines Flugzeugs.

\begin{figure}[h]
%\centering
\begin{subfigure}{0.4\textwidth}
  \centering
  \begin{overpic}[width=1\linewidth]{./Bilder/KSFliegerSeitlich.png}
		\put(47,30){$\underline{p}_\mathrm{sp}^\mathcal{F}$}
		\put(25,33){$\theta$}
		\put(31,31){$\alpha$}
		\put(-1,46){$x_\mathrm{f}$}
		\put(-4,34){$x_\mathrm{a}$}
		\put(-4,28){$x_\mathrm{g}$}
		\put(46,1){$z_\mathrm{g}$}
		\put(41,1){$z_\mathrm{a}$}
		\put(34,2){$z_\mathrm{f}$}
	\end{overpic}
  \caption{Neigungswinkel ($\theta$) und Anstellwinkel($\alpha$). Seitliche Sicht.}
  %\label{fig:sub1}
\end{subfigure}%
\hspace{3cm}
\begin{subfigure}{0.4\textwidth}
  \centering
   \begin{overpic}[width=1\linewidth]{./Bilder/KSFliegerOben.png}
	    \put(41,30){$\underline{p}_\mathrm{sp}^\mathcal{F}$}
		\put(20,20){$\psi$}
		\put(26,26){$\beta$}
		\put(-2,6){$x_\mathrm{f}$}
		\put(-4.5,18){$x_\mathrm{a}$}
		\put(-3,27){$x_\mathrm{g}$}
		\put(37,53){$y_\mathrm{g}$}
		\put(28,51){$y_\mathrm{a}$}
		\put(23,46){$y_\mathrm{f}$}
	\end{overpic}
  \caption{Gierwinkel ($\psi$) und Schiebewinkel($\beta$).Sicht von oben.}
  %\label{fig:sub2}
\end{subfigure}\\
\begin{subfigure}{0.5\textwidth}
  \begin{center}
  
   \begin{overpic}[width=1\linewidth]{./Bilder/KSFliegerVorne.png}
		\put(52,25){$\underline{p}_\mathrm{sp}^\mathcal{F}$}
		\put(35,22){$\phi$}
		\put(20,4.5){$y_\mathrm{f}$}
		\put(18,13.5){$y_\mathrm{a}$}
		\put(17,21.5){$y_\mathrm{g}$}
		\put(50,6){$z_\mathrm{g}$}
		\put(54,6.5){$z_\mathrm{a}$}
		\put(58,7){$z_\mathrm{f}$}
		
	
	\end{overpic}
  \caption{Rollwinkel ($\phi$). Frontale Sicht.}
  %\label{fig:sub2}
  \end{center}
\end{subfigure}
\caption{Darstellung der drei behandelten Koordinatensystemen und deren Anwinklung}
\label{fig:KoordSyst}
\end{figure}
 
\subsection{Koordinatentransformationen}
Damit die Modellbildung einheitlich über ein bestimmtes Koordinatensystem erfolgt, muss man über die Werkzeuge verfügen, die das Bezugssystem transformieren können. Dabei nutz man die aerodynamischen Winkel ($\alpha$, $\beta$) und die Eulerwinkel ($\phi$,$\theta$,$\psi$), die auf Abbildung \ref{fig:KoordSyst} definiert sind. Die Autoren von \cite{FlugmechanikBuch} und \cite{Fichter2020} empfehlen folgende Darstellung.\\

\begin{itemize}
\item Transformation von aerodynamischem zu körperfestem Bezugssystem. 
\begin{equation}
\textbf{T}_\mathrm{fa} = \begin{bmatrix} 
\cos(\alpha) & 0& -\sin(\alpha)\\
0 & 1 & 0\\
\sin(\alpha)&  0& \cos(\alpha)
\end{bmatrix}\begin{bmatrix} 
\cos(\beta) & -\sin(\beta)& 0\\
\sin(\beta) & \cos(\beta) & 0\\
0&  0& 1
\end{bmatrix}
\end{equation}
\item Transformation von körperfestem zu aerodynamischem Bezugssystem. 
\begin{equation}
\textbf{T}_\mathrm{af} = \textbf{T}_\mathrm{fa}^T
\end{equation}
\item Transformation von geodätischem zu körperfestem Bezugssystem. 
\begin{equation}
\textbf{T}_\mathrm{fg} =\begin{bmatrix} 
1 & 0& 0\\
0 & \cos(\phi) & \sin(\phi)\\
0&  -\sin(\phi)& \cos(\phi)
\end{bmatrix} \begin{bmatrix} 
\cos(\theta) & 0& -\sin(\theta)\\
0 & 1 & 0\\
\sin(\theta)&  0& \cos(\theta)
\end{bmatrix}\begin{bmatrix} 
\cos(\psi) & \sin(\psi)& 0\\
-\sin(\psi) & \cos(\psi) & 0\\
0&  0& 1
\end{bmatrix}
\end{equation}
\item Transformation von körperfestem zu geodätischem Bezugssystem. 
\begin{equation}
\textbf{T}_\mathrm{gf} = \textbf{T}_\mathrm{fg}^T
\end{equation}
\end{itemize}

Folgendes Beispiel zeigt wie einer Geschwindigkeitsvektor in körperfestem Koordinatensystem auf ein geodätisches Bezugssystem gebracht werden kann.\\
\begin{equation*}
\underline{v}^\mathcal{G} = \textbf{T}_\mathrm{gf}\underline{v}^\mathcal{F} = \textbf{T}_\mathrm{gf}\underline{v}
\end{equation*}
Für die Herleitung der Bewegungsgleichungen wird noch das \textit{erdfeste} Bezugssystem (Index $\mathcal{E}$) angewandt. Der Unterschied zwischen dem erdfesten und geodätischen KS liegt an dem Mittelpunkt, welcher sich bei dem erdfesten KS  auf der Erdoberfläche befindet. Das erdfeste KS wird außerdem für die Berechnung der Flugzeugposition bezüglich der Erdoberfläche verwendet. Die Matrix $\textbf{T}_\mathrm{fe}(\lambda,\gamma) $ repräsentiert die Transformation von erdfesten zu körperfesten Koordinaten, wobei $\lambda$ den Bahnazimutwinkel und $\gamma$ den Bahnneigungswinkel darstellen. Deren Berechnung wird im Kapitel \ref{cha:Modellbildung} vorgestellt.