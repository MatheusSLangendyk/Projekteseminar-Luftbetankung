% Das Verzeichnis iatsada zum Pfad hinzufügen, so dass \usepackage{iatsada} und die Bib-Stile funktionieren
% Sofern dies nicht vorhanden ist, wird auf das gemeinsame iat_sada_common/texmf/tex/latex/iatsada/ zurückgegriffen
% https://tex.stackexchange.com/questions/79058/can-a-default-path-be-set-globally-for-input-akin-to-graphicspath
\makeatletter
\providecommand*{\input@path}{}
\edef\input@path{{iatsada/}{../iat_sada_common/texmf/tex/latex/iatsada/}{../iat_sada_common/texmf/tex/latex/iatsada/common/logos/}{../iat_sada_common/texmf/tex/latex/biblatex-iatsada/}{../iat_sada_common/texmf/tex/latex/biblatex-iatsada/bbx/}{../iat_sada_common/texmf/tex/latex/biblatex-iatsada/cbx/}{../iat_sada_common/texmf/tex/latex/abk/}{../iat_sada_common/texmf/tex/latex/ifclass/}{../iat_sada_common/texmf/tex/latex/uniinput/}\input@path}% prepend
\makeatother

% =================================================================================
% Hier auswählen, ob Abgabeversion für TUbama oder nicht
% =================================================================================
\newif\ifSADATUbama
\SADATUbamatrue					% Abgabeversion für TUbama (PDF/A)
\SADATUbamafalse				% keine Abgabeversion

\newcommand{\selectMainLanguage}{\selectlanguage{ngerman}} % Wenn die Arbeit auf deutsch geschrieben wird
%\newcommand{\selectMainLanguage}{\selectlanguage{english}} % Wenn die Arbeit auf englisch geschrieben wird

% =================================================================================
% Ausnahmen von der automatischen Silbentrennung
% =================================================================================
\hyphenation{Aktu-ali-sie-rung Screen-shots Pa-rallel-ro-bo-ter Zu-stands-raum-mo-del-le nach-voll-zieh-bar Pro-jekt-se-mi-nar}
% =================================================================================

% =================================================================================
% Hier auswählen, ob TUD-Design oder nicht
% =================================================================================
\newif\ifTUDdesign
\TUDdesigntrue					% TUD-Design
%\TUDdesignfalse				% Für Rechner ohne installierte TUDdesign-Pakete
% =================================================================================

% =================================================================================
% Hier NICHTS ändern!
% =================================================================================
\ifSADATUbama
	% Replace the following information with your document's actual
% metadata. If you do not want to set a value for a certain parameter,
% just omit it.
%
% Symbols permitted in metadata
% =============================
% 
% Within the metadata, all printable ASCII characters except
% '\', '{', '}', and '%' represent themselves. Also, all printable
% Unicode characters from the basic multilingual plane (i.e., up to
% code point U+FFFF) can be used directly with the UTF-8 encoding. 
% Consecutive whitespace characters are combined into a single
% space. Whitespace after a macro such as \copyright, \backslash, or
% \sep is ignored. Blank lines are not permitted. Moreover, the
% following markup can be used:
%
%  '\ '         - a literal space  (for example after a macro)                  
%   \%          - a literal '%'                                                 
%   \{          - a literal '{'                                                 
%   \}          - a literal '}'                                                 
%   \backslash  - a literal '\'                                                 
%   \copyright  - the (c) copyright symbol                                      
%
% The macro \sep is only permitted within \Author, \Keywords, and
% \Org.  It is used to separate multiple authors, keywords, etc.
% 
% List of supported metadata fields
% =================================
% 
% Here is a complete list of user-definable metadata fields currently
% supported, and their meanings. More may be added in the future.
% 
% General information:
%
%  \Author           - the document's human author. Separate multiple
%                      authors with \sep.
%  \Title            - the document's title.
%  \Keywords         - list of keywords, separated with \sep.
%  \Subject          - the abstract. 
%  \Org              - publishers.
% 
% Copyright information:
%
%  \Copyright        - a copyright statement.
%  \CopyrightURL     - location of a web page describing the owner
%                      and/or rights statement for this document.
%  \Copyrighted      - 'True' if the document is copyrighted, and
%                      'False' if it isn't. This is automatically set
%                      to 'True' if either \Copyright or \CopyrightURL
%                      is specified, but can be overridden. For
%                      example, if the copyright statement is "Public
%                      Domain", this should be set to 'False'.
%
% Publication information:
%
% \PublicationType   - The type of publication. If defined, must be
%                      one of book, catalog, feed, journal, magazine,
%                      manual, newsletter, pamphlet. This is
%                      automatically set to "journal" if \Journaltitle
%                      is specified, but can be overridden.
% \Journaltitle      - The title of the journal in which the document
%                      was published. 
% \Journalnumber     - The ISSN for the publication in which the
%                      document was published.
% \Volume            - Journal volume.
% \Issue             - Journal issue/number.
% \Firstpage         - First page number of the published version of
%                      the document.
% \Lastpage          - Last page number of the published version of
%                      the document.
% \Doi               - Digital Object Identifier (DOI) for the
%                      document, without the leading "doi:".
% \CoverDisplayDate  - Date on the cover of the journal issue, as a
%                      human-readable text string.
% \CoverDate         - Date on the cover of the journal issue, in a
%                      format suitable for storing in a database field
%                      with a 'date' data type.
\begin{filecontents*}{\jobname.xmpdata}
	\Title{Eine LaTeX-Vorlage für schriftliche (Abschluss-)Arbeiten am IAT}
	
	\Author{Martin Mustermann \sep Erika Musterfrau \sep John Doe}
	
	\Copyright{Copyright \copyright\ 2018 "Martin Mustermann, Erika Musterfrau, John Doe"}
	
	\Keywords{Studienarbeit \sep Bachelorarbeit \sep Masterarbeit \sep Diplomarbeit \sep Vorlage \sep LaTeX Klasse}
	
	\Subject{Das LaTeX-Dokument sada_tudreport ist eine Vorlage für schriftliche Arbeiten (Proseminar-, Projektseminar-, Studien-, Bachelor-, Master- und Diplomarbeiten, etc.) am Institut für Automatisierungstechnik der TU Darmstadt. Das Layout ist an die Richtlinien zur Anfertigung von Studien- und Diplomarbeiten angepasst und durch Modifikation der Klasse tudreport realisiert, so dass in der Arbeit die gewohnten LaTeX-Befehle benutzt werden können. Die vorliegende Anleitung beschreibt die Klasse und gibt grundlegende Hinweise zum Verfassen wissenschaftlicher Arbeiten. Sie ist außerdem ein Beispiel für den Aufbau einer Studien-, Bachelor-, Master- bzw. Diplomarbeit.}
\end{filecontents*}
\fi
\ifTUDdesign
	% Ggf. Option "numbers=noenddot" einfügen, damit "Abbildung 3.1: _" statt "Abbildung 3.1.: _" verwendet wird.
	\documentclass[11pt, twoside, colorback, accentcolor=tud2c, nopartpage, bigchapter, fleqn, ngerman, longdoc]{tudreport}
\else
	\documentclass[11pt, a4paper, twoside, fleqn, ngerman]{scrreprt}% Für Entwurf auf Rechnern ohne installierte TUDdesign-Pakete
	\usepackage{exscale}% Korrektur math-Zeichen
	\usepackage{eurosym}
\fi
% Dieses File dient zum Einbinden wichtiger und nützlicher Pakete.
% Nicht alle Pakete müssen verwendet werden.
%
\usepackage[T1]{fontenc}	% Für Silbentrennung bei Wörten mit Sonderzeichen (z.B. Umlaute)
\usepackage[utf8]{inputenc}	% Um Sonderzeichen (ä, ß, é, ...) direkt eingeben zu können
\usepackage[english,main=ngerman]{babel}
							% Für Sprachenspezifisches
							% ngerman ist schon als globale Option definiert
\usepackage{ifclass}
\usepackage{iflang}			% für sprachabhängige Einstellungen
\usepackage{xkvltxp}		% für Macros in key-value Paketoptionen

%\usepackage{helvet}		% Helvetica als Standard-Sans-Schriftart
\usepackage[stable]{footmisc}
\usepackage{booktabs}


\usepackage{graphicx}		% zum Einbinden von Postscript
\usepackage{psfrag}			% Beschriftung der Bilder
\ifbeamer
	\usepackage{amsmath}% Mehr mathematischer Formelsatz
\else
	\usepackage[tbtags]{amsmath}% Mehr mathematischer Formelsatz
\fi
%\usepackage{amssymb}		% Mehr mathematische Symbole
%\usepackage{amsthm}

\usepackage{float}			% Für Parameter [H] bei Fließobjekten

\usepackage{epsfig}			% Um eps-Bilder einzubinden
\usepackage{scrhack}		% Um Warnung "float@addtolists detected" zu unterdrücken 
\usepackage{subcaption}		% Für Unterabbildungen
\usepackage{ltxtable} 		% Vereinigt TabularX und Longtable
\usepackage{upgreek}		% Für nicht-kursive kleine griechischen Buchstaben
\usepackage{uniinput}		% Diverse Makros zum Ersetzen von UTF8 Zeichen durch Latex äquivalente					
\usepackage{multirow}		% Für mehrzeilige Felder in Tabellen
\usepackage{rotating}		% Zum Drehen von Objekten
\usepackage{icomma}			% Damit nach Dezimalkommas kein Abstand eingefügt wird
\usepackage{dcolumn}		% Zum Ausrichten von Tabellenspalten am Dezimaltrennzeichen

%\usepackage{wrapfig}		% Für kleine Bilder am Rand
%\usepackage{floatflt}		% Alternative zu wrapfig
%\usepackage[hang]{caption}	% Damit mehrzeilige Bildunterschriften gut aussehen
\usepackage{textcomp}		% Für Sonderzeichen im normalen Text
							% (offensichtlich in tudreport schon eingebunden)
\usepackage[ngerman]{varioref}		% Für vref
\usepackage{color}			% Für farbigen Text
\usepackage{placeins}		% Für \FloatBarrier
\usepackage{xspace}
\usepackage{cancel}			% Zum Wegstreichen von Gleichungstermen \cancel{a} oder \cancelto{0}{a}
\usepackage{listings}		% Um formatierten Quellcode einzubinden


\usepackage{array}			% Für Zellentyp "m{}" in tabular-Umgebungen (Vertikal zentriert)
\usepackage{moreverb}		% Für Umgebung "`verbatimtab"' (Verbatim mit Tabs)
\renewcommand{\verbatimtabsize}{4\relax}	% Standardtabweite in "`verbatimtab"'
											% ist 4 Zeichen

\usepackage{abk}

\usepackage{siunitx}		% Um Zahlen mit Einheiten korrekt darzustellen \SI{12}{\meter\per\second} 
\sisetup{
	range-units=single,
	range-phrase={\,\textnormal{\textendash}\,},
	output-decimal-marker={,},
	%per-mode=fraction,
	per-mode=reciprocal,
	per-mode=symbol,
	math-micro=\muup,% Im TU-Design passt das µ sonst nicht
	math-ohm  =\Omegaup,
	text-micro={\fontfamily{mdbch}\textmu},
	text-ohm  ={\fontfamily{mdbch}\textohm}
}

% ======== Glossar =========
\newif\ifSADAuseglossaries
\SADAuseglossariestrue
\SADAuseglossariesfalse
\ifSADAuseglossaries
	\usepackage[automake, acronym, symbols, toc, translate=babel]{glossaries}
	\addto\captionsngerman{%
		\renewcommand*{\acronymname}{Abkürzungsverzeichnis}
		\renewcommand*{\glssymbolsgroupname}{Symbolverzeichnis}
	}
\fi

% ======== Anfang Literaturverwaltung =====
%\usepackage[datename]{babelbib}	% Für deutsche Literaturverwaltung
%\setbtxfallbacklanguage{ngerman}
%\btxprintISSN{false}
%\setbibliographyfont{lastname}{\textsc}%
\usepackage[babel]{csquotes}
\ifbeamer
	\usepackage[
		backend=biber,
		style=iatsada-numeric,
		backref=false
	]{biblatex}					% Für deutsche Literaturverwaltung
\else
	\usepackage[
		backend=biber,
		style=iatsada-numeric,
		backref=true
	]{biblatex}					% Für deutsche Literaturverwaltung
	% ==========================================
\fi

\ifbeamer
	% Hier Pakete die nur für Präsentationen benötigt werden
	\usepackage{animate}
	\usepackage{multimedia}
\fi

\ifbeamer
	\usepackage{todonotes}		% Für Anmerkungen \todo{Bitte ändern} bzw. \todo[inline]{Bitte ändern}
\else
	\usepackage[textwidth=1.8cm, textsize=tiny]{todonotes}		% Für Anmerkungen \todo{Bitte ändern} bzw. \todo[inline]{Bitte ändern}	
	%muss eigentlich hinter begin{document} stehen, macht dann aber die Fusszeile kaputt
	%\setlength{\marginparwidth}{1.8cm} % notwendig damit todos am Rand korrekt dargestellt werden (siehe Doku todonotes Paket)
\fi

% Das Packet hyperref immer als letztes einbinden (nur bookmark darf danach kommen)!
% Für Verlinkungen im erzeugten pdf
\ifbeamer
	\usepackage[colorlinks=false]{hyperref}
\else
	\usepackage[unicode=true, colorlinks=false, breaklinks=true]{hyperref}
	\ifSADATUbama
		\usepackage[a-1b]{pdfx}
	\fi
\fi
\usepackage{bookmark}			% verwendete Pakete einbinden
\usepackage{fp} 
\usepackage{tikz}				% Zum Erzeugen von Bildern mit TikZ
\usepackage{pgf}
\usepackage{pgfplots}			% Zum Erzeugen von Diagrammen mit pgfplots
\usepackage{pgfplotstable}
\usetikzlibrary{arrows}
\usetikzlibrary{arrows.meta}
\usetikzlibrary{backgrounds}
\usetikzlibrary{calc}
\usetikzlibrary{circuits}
\usetikzlibrary{circuits.ee.IEC}
\usetikzlibrary{decorations.markings}
\usetikzlibrary{decorations.shapes}
\usetikzlibrary{decorations.text}
\usetikzlibrary{fadings}
\usetikzlibrary{fit}
\usetikzlibrary{intersections}
\usetikzlibrary{matrix}
\usetikzlibrary{patterns}
\usetikzlibrary{positioning}
\usetikzlibrary{shapes}
\usetikzlibrary{shadows}
\usetikzlibrary{spy}
\IfLanguageName{german}{
	\tikzset{
		/pgf/number format/use comma,
		/pgf/number format/1000 sep=\,
	}
}{}
\IfLanguageName{ngerman}{
	\tikzset{
		/pgf/number format/use comma,
		/pgf/number format/1000 sep=\,
	}
}{}
\usepgfplotslibrary{groupplots}
%\usepgfplotslibrary{fillbetween}

\pgfplotsset{compat=newest}
\IfLanguageName{german}{
	\pgfplotsset{
		x tick label style={/pgf/number format/use comma, /pgf/number format/1000 sep=\,},
		y tick label style={/pgf/number format/use comma, /pgf/number format/1000 sep=\,},
		z tick label style={/pgf/number format/use comma, /pgf/number format/1000 sep=\,}
	}
}{}
\IfLanguageName{ngerman}{
	\pgfplotsset{
		x tick label style={/pgf/number format/use comma, /pgf/number format/1000 sep=\,},
		y tick label style={/pgf/number format/use comma, /pgf/number format/1000 sep=\,},
		z tick label style={/pgf/number format/use comma, /pgf/number format/1000 sep=\,}
	}
}{}

\edef\pgfdatafolder{./Bilder/Daten} % Verzeichnis in dem die csv Dateien für pgfplots liegen

\newcommand{\xmin}{1e-2}
\newcommand{\xmax}{1e2}
\newcommand{\mywidth}{0.8\textwidth}
\newcommand{\myheight}{60mm}

\newcommand{\omegaD}{1}

\newcommand{\bodestyle}{
	\pgfplotsset{
		major grid style={line width=0.3pt, color=gray},
		minor grid style={line width=0.3pt, color=gray},
		major tick style={line width=0.4pt, color=black},
		major tick length={4pt},
		minor tick length={3pt},
		tick label style={font=\small}
	}
}

\newcommand{\nyquiststyle}{
	\pgfplotsset{
		major grid style={line width=0.3pt, color=gray},
		minor grid style={line width=0.3pt, color=gray},
		major tick style={line width=0.4pt, color=black},
		major tick length={4pt},
		minor tick length={3pt},
		tick label style={font=\small}
	}
}

\newcommand{\plotstyle}{
	\pgfplotsset{
		major grid style={line width=0.3pt, color=gray},
		minor grid style={line width=0.3pt, color=gray},
		major tick style={line width=0.4pt, color=black},
		major tick length={4pt},
		minor tick length={3pt},
		tick label style={font=\small}
	}
}

\newcommand{\plotyystyle}{
	\pgfplotsset{
		every non boxed y axis/.style={ytick align=inside,y axis line style={-}},
		every boxed y axis/.style={}
	}
}

\newenvironment{bodeAmpDB}[1][]{
\bodestyle
\begin{semilogxaxis}[
	ylabel=$|G(\mathrm{j}\omega)|_{\mathrm{dB}}$,
	%ylabel=Amplitude in dB,
	ylabel style={yshift=2pt},
	enlarge x limits=false,
	xminorgrids=true,
	xmajorgrids=true,
	ymajorgrids=true,
	yminorgrids=true,
	xticklabels=\empty,
	width=\mywidth,
	height=\myheight,#1]
}{\end{semilogxaxis}}

\newenvironment{bodeAmpLOG}[1][]{
\begin{loglogaxis}[
	ylabel=$|G(\mathrm{j}\omega)|$,
	%ylabel=Amplitude,
	ylabel style={yshift=2pt},
	enlarge x limits=false,
	xminorgrids=true,
	xmajorgrids=true,
	ymajorgrids=true,
	yminorgrids=true,
	xticklabels=\empty,
	y tick label style={font=\small},
	width=\mywidth,
	height=\myheight,#1]
}{\end{loglogaxis}}

\newenvironment{bodePhase}[1][]{
\begin{scope}[yshift=-\myheight+12mm]	% zweiten Plot (Phase) unter den Amplitudengang setzen
	\bodestyle
	\begin{semilogxaxis}[
		xlabel=Frequenz $\omega$ in \ensuremath{\mathrm{{\frac{rad}{s}}}},
		%ylabel=Phase in $^\circ$,
		ylabel=$\angle{G(\mathrm{j}\omega)}$,
		ylabel style={yshift=2pt},
		enlarge x limits=false,
		xminorgrids=true,
		xmajorgrids=true,
		ymajorgrids=true,
		yminorgrids=true,
		ytick={-360, -315,..., 360},
		yticklabel={$\pgfmathprintnumber{\tick}^\circ$},% ° als Einheitenzeichen an alle yticks
		width=\mywidth,
		height=\myheight,#1]
}{\end{semilogxaxis}\end{scope}}

\newenvironment{plotyyLeft}[1][]{
\plotyystyle
\begin{axis}[
	scale only axis,
	enlarge x limits=0,
	axis y line=left,
	width=\mywidth,
	height=\myheight,#1]
}{\end{axis}}

\newenvironment{plotyyRight}[1][]{
\plotyystyle
\begin{axis}[
	scale only axis,
	enlarge x limits=0,
	axis y line=right,
	axis x line=none,
	width=\mywidth,
	height=\myheight,#1]
}{\end{axis}}

\newcommand{\TikZpole}[1][blue]{
\begin{tikzpicture}
	\draw[#1, very thick, line cap=round] (-1,1)  -- (1,-1);
	\draw[#1, very thick, line cap=round] (-1,-1) -- (1,1);
\end{tikzpicture}
}

\newcommand{\TikZzero}[1][blue]{
\begin{tikzpicture}
	\draw[#1, very thick] (0,0) circle (5pt);
\end{tikzpicture}
}

% Blockschaltbilder
\newcommand{\TikZscale}{1}
\newcommand{\mm}{*\TikZscale mm}
% TikZstyles f�r Blockschaltbilder

\renewcommand{\TikZscale}{1}
\tikzset{every picture/.style={node distance=4\mm, >=stealth'}}

%% Bl�cke
	% Rechteckige Bl�cke
	\tikzstyle{block}      = [draw, semithick, rectangle, minimum height=8\mm, minimum width=8\mm, inner sep=3pt]
	\tikzstyle{NLblock}    = [draw, semithick, rectangle, minimum height=8\mm, minimum width=8\mm, inner sep=3pt, double distance=1.2pt]
	\tikzstyle{PICblock}   = [draw, semithick, rectangle, minimum height=8\mm, minimum width=8\mm, inner sep=2pt]
	\tikzstyle{NLPICblock} = [draw, semithick, rectangle, minimum height=8\mm, minimum width=8\mm, inner sep=3pt, double distance=1.2pt]
	\tikzstyle{noblock}	   = [rectangle, inner sep=-0.6pt]

	% Dreieckige Bl�cke
	\tikzstyle{Rgain}			 = [draw, semithick, isosceles triangle, inner sep=1pt, minimum height=8\mm, isosceles triangle apex angle=60]
	\tikzstyle{Lgain}			 = [draw, semithick, isosceles triangle, inner sep=1pt, minimum height=8\mm, isosceles triangle apex angle=60, shape border rotate=180]
	\tikzstyle{Ugain}			 = [draw, semithick, isosceles triangle, inner sep=1pt, minimum height=8\mm, isosceles triangle apex angle=60, shape border rotate=90]
	\tikzstyle{Dgain}			 = [draw, semithick, isosceles triangle, inner sep=1pt, minimum height=8\mm, isosceles triangle apex angle=60, shape border rotate=-90]

	% Runde Bl�cke
	\tikzstyle{sum}   	   = [draw, semithick, circle, inner sep=1pt, minimum size=3\mm]
	\tikzstyle{branch}		 = [draw, circle, inner sep=0pt, minimum size=1\mm, fill=black]
	\tikzstyle{BRANCH}		 = [coordinate]


%% Verbindungselemente
	% Linien mit Pfeil
	\tikzstyle{to}  		= [->, thick]
	\tikzstyle{toNL}		= [->, thick, shorten >=0.9pt]
	\tikzstyle{NLto}		= [->, thick, shorten <=0.9pt]
	\tikzstyle{NLtoNL}	= [->, thick, shorten <=0.9pt, shorten >=0.9pt]
	
	\tikzstyle{TO}  		= [semithick, double distance=2pt, shorten >=2mm, decoration={markings,mark=at position 1 with {\arrow[semithick]{open triangle 60}}}, preaction={decorate},postaction={draw, line width=2pt, white, shorten >= 1.5mm}]
	
	\tikzstyle{TONL}  		= [semithick, double distance=2pt, shorten >=2.2mm, decoration={markings,mark=at position 1 with {\arrow[semithick]{open triangle 60}}, transform={xshift=-0.7pt}}, preaction={decorate},postaction={draw, line width=2pt, white, shorten >= 1.5mm}]
	
	\tikzstyle{NLTO}  		= [semithick, double distance=2pt, shorten <=0.9pt, shorten >=2mm, decoration={markings,mark=at position 1 with {\arrow[semithick]{open triangle 60}}}, preaction={decorate}, postaction={draw, line width=2pt, white, shorten >= 1.5mm}]
	
	\tikzstyle{NLTONL}  		= [semithick, double distance=2pt, shorten <=0.9pt, shorten >=2.2mm, decoration={markings,mark=at position 1 with {\arrow[semithick]{open triangle 60}}}, preaction={decorate},postaction={draw, line width=2pt, white, shorten >= 1.5mm}]
	
	\tikzstyle{innerWhite} = [semithick, white,line width=2pt, shorten >= 2mm, shorten <= 2mm]
	
	%\tikzstyle{TO}  		= [semithick, double distance=2pt, shorten >=2mm, decoration={markings,mark=at position 1 with {\arrow[semithick]{open triangle 60}}}, postaction={decorate}]
		%
	%\tikzstyle{TONL} = [semithick, double distance=2pt, shorten >=2.2mm, decoration={markings,mark=at position 1 with {\arrow[semithick]{open triangle 60}}, transform={xshift=-0.7pt}}, postaction={decorate}]
	%
	%\tikzstyle{NLTO} = [semithick, double distance=2pt, shorten <=0.79pt, shorten >=2mm, decoration={markings,mark=at position 1 with {\arrow[semithick]{open triangle 60}}}, postaction={decorate}]
	%
	%\tikzstyle{NLTONL} = [semithick, double distance=2pt, shorten <=0.79pt, shorten >=2.2mm, decoration={markings,mark=at position 1 with {\arrow[semithick]{open triangle 60}}, transform={xshift=-0.7pt}}, postaction={decorate}]
	

	% Linien ohne Pfeil 
	\tikzstyle{line}       = [thick]
	\tikzstyle{lineNL}     = [thick, shorten >= 0.6pt]
	\tikzstyle{NLline}     = [thick, shorten <= 0.6pt]
	\tikzstyle{NLlineNL}	 = [thick, shorten <= 0.6pt, shorten >= 0.6pt]
	
	% Doppellinien ohne Pfeil
	\tikzstyle{LINE}       = [semithick, double distance=2pt]
	\tikzstyle{LINENL}     = [semithick, double distance=2pt, shorten >= 0.9pt]
	\tikzstyle{NLLINE}     = [semithick, double distance=2pt, shorten <= 0.9pt]
	\tikzstyle{NLLINENL}	 = [semithick, double distance=2pt, shorten <= 0.9pt, shorten >= 0.9pt]
	
	\tikzstyle{neg}				 = [postaction={decorate,decoration={markings, mark=at position 1 with{\draw[-](-2pt,-3pt)--(-2pt,-7pt);}}}]
	
	\tikzstyle{labelabove} = [above, anchor=base, yshift=0.75ex]
	\tikzstyle{labelbelow} = [below, anchor=base, yshift=-2ex]
	\tikzstyle{labelright} = [right]
	\tikzstyle{labelleft}  = [left]
	
	
	% Pins und Labels
	\tikzstyle{every pin edge}	= [<-, thick]
	\tikzstyle{every pin}	 			= [pin distance=5\mm]
	\tikzstyle{every label}			= [font=\small]
	\tikzstyle{terminal}				= [coordinate]
	
	% extra Symbole f�r circuits-Library
	\tikzstyle{jack}	= [draw, circle, minimum size=1.5mm, inner sep=0]


\newcommand{\BSBnormal}[1]{
	% TikZstyles f�r Blockschaltbilder

\renewcommand{\TikZscale}{1}
\tikzset{every picture/.style={node distance=4\mm, >=stealth'}}

%% Bl�cke
	% Rechteckige Bl�cke
	\tikzstyle{block}      = [draw, semithick, rectangle, minimum height=8\mm, minimum width=8\mm, inner sep=3pt]
	\tikzstyle{NLblock}    = [draw, semithick, rectangle, minimum height=8\mm, minimum width=8\mm, inner sep=3pt, double distance=1.2pt]
	\tikzstyle{PICblock}   = [draw, semithick, rectangle, minimum height=8\mm, minimum width=8\mm, inner sep=2pt]
	\tikzstyle{NLPICblock} = [draw, semithick, rectangle, minimum height=8\mm, minimum width=8\mm, inner sep=3pt, double distance=1.2pt]
	\tikzstyle{noblock}	   = [rectangle, inner sep=-0.6pt]

	% Dreieckige Bl�cke
	\tikzstyle{Rgain}			 = [draw, semithick, isosceles triangle, inner sep=1pt, minimum height=8\mm, isosceles triangle apex angle=60]
	\tikzstyle{Lgain}			 = [draw, semithick, isosceles triangle, inner sep=1pt, minimum height=8\mm, isosceles triangle apex angle=60, shape border rotate=180]
	\tikzstyle{Ugain}			 = [draw, semithick, isosceles triangle, inner sep=1pt, minimum height=8\mm, isosceles triangle apex angle=60, shape border rotate=90]
	\tikzstyle{Dgain}			 = [draw, semithick, isosceles triangle, inner sep=1pt, minimum height=8\mm, isosceles triangle apex angle=60, shape border rotate=-90]

	% Runde Bl�cke
	\tikzstyle{sum}   	   = [draw, semithick, circle, inner sep=1pt, minimum size=3\mm]
	\tikzstyle{branch}		 = [draw, circle, inner sep=0pt, minimum size=1\mm, fill=black]
	\tikzstyle{BRANCH}		 = [coordinate]


%% Verbindungselemente
	% Linien mit Pfeil
	\tikzstyle{to}  		= [->, thick]
	\tikzstyle{toNL}		= [->, thick, shorten >=0.9pt]
	\tikzstyle{NLto}		= [->, thick, shorten <=0.9pt]
	\tikzstyle{NLtoNL}	= [->, thick, shorten <=0.9pt, shorten >=0.9pt]
	
	\tikzstyle{TO}  		= [semithick, double distance=2pt, shorten >=2mm, decoration={markings,mark=at position 1 with {\arrow[semithick]{open triangle 60}}}, preaction={decorate},postaction={draw, line width=2pt, white, shorten >= 1.5mm}]
	
	\tikzstyle{TONL}  		= [semithick, double distance=2pt, shorten >=2.2mm, decoration={markings,mark=at position 1 with {\arrow[semithick]{open triangle 60}}, transform={xshift=-0.7pt}}, preaction={decorate},postaction={draw, line width=2pt, white, shorten >= 1.5mm}]
	
	\tikzstyle{NLTO}  		= [semithick, double distance=2pt, shorten <=0.9pt, shorten >=2mm, decoration={markings,mark=at position 1 with {\arrow[semithick]{open triangle 60}}}, preaction={decorate}, postaction={draw, line width=2pt, white, shorten >= 1.5mm}]
	
	\tikzstyle{NLTONL}  		= [semithick, double distance=2pt, shorten <=0.9pt, shorten >=2.2mm, decoration={markings,mark=at position 1 with {\arrow[semithick]{open triangle 60}}}, preaction={decorate},postaction={draw, line width=2pt, white, shorten >= 1.5mm}]
	
	\tikzstyle{innerWhite} = [semithick, white,line width=2pt, shorten >= 2mm, shorten <= 2mm]
	
	%\tikzstyle{TO}  		= [semithick, double distance=2pt, shorten >=2mm, decoration={markings,mark=at position 1 with {\arrow[semithick]{open triangle 60}}}, postaction={decorate}]
		%
	%\tikzstyle{TONL} = [semithick, double distance=2pt, shorten >=2.2mm, decoration={markings,mark=at position 1 with {\arrow[semithick]{open triangle 60}}, transform={xshift=-0.7pt}}, postaction={decorate}]
	%
	%\tikzstyle{NLTO} = [semithick, double distance=2pt, shorten <=0.79pt, shorten >=2mm, decoration={markings,mark=at position 1 with {\arrow[semithick]{open triangle 60}}}, postaction={decorate}]
	%
	%\tikzstyle{NLTONL} = [semithick, double distance=2pt, shorten <=0.79pt, shorten >=2.2mm, decoration={markings,mark=at position 1 with {\arrow[semithick]{open triangle 60}}, transform={xshift=-0.7pt}}, postaction={decorate}]
	

	% Linien ohne Pfeil 
	\tikzstyle{line}       = [thick]
	\tikzstyle{lineNL}     = [thick, shorten >= 0.6pt]
	\tikzstyle{NLline}     = [thick, shorten <= 0.6pt]
	\tikzstyle{NLlineNL}	 = [thick, shorten <= 0.6pt, shorten >= 0.6pt]
	
	% Doppellinien ohne Pfeil
	\tikzstyle{LINE}       = [semithick, double distance=2pt]
	\tikzstyle{LINENL}     = [semithick, double distance=2pt, shorten >= 0.9pt]
	\tikzstyle{NLLINE}     = [semithick, double distance=2pt, shorten <= 0.9pt]
	\tikzstyle{NLLINENL}	 = [semithick, double distance=2pt, shorten <= 0.9pt, shorten >= 0.9pt]
	
	\tikzstyle{neg}				 = [postaction={decorate,decoration={markings, mark=at position 1 with{\draw[-](-2pt,-3pt)--(-2pt,-7pt);}}}]
	
	\tikzstyle{labelabove} = [above, anchor=base, yshift=0.75ex]
	\tikzstyle{labelbelow} = [below, anchor=base, yshift=-2ex]
	\tikzstyle{labelright} = [right]
	\tikzstyle{labelleft}  = [left]
	
	
	% Pins und Labels
	\tikzstyle{every pin edge}	= [<-, thick]
	\tikzstyle{every pin}	 			= [pin distance=5\mm]
	\tikzstyle{every label}			= [font=\small]
	\tikzstyle{terminal}				= [coordinate]
	
	% extra Symbole f�r circuits-Library
	\tikzstyle{jack}	= [draw, circle, minimum size=1.5mm, inner sep=0]

	\input{#1}
}			% Alles was mit tikz und pgf zu tun hat
% Diese Datei dient zum definieren nützlicher Befehle.
% Sie soll lediglich als Beispiel dienen, wie Befehle definiert werden, und welche Befehle nützlich sein können.
\newcommand{\eqp}{\ensuremath{\, \, \, .}}
\newcommand{\M}[1]{\textbf{#1}} %Matrix  M
\newcommand{\Mr}[1]{\textbf{#1}_\rho} %Matrix mit tiefgestelltem rho, M_rho
\newcommand{\Mtil}[1]{\tilde{\textbf{#1}}} %Matrix mit Tilde
\newcommand{\Mtilt}[2]{\tilde{\textbf{#1}}_{#2}} %Matrix mit Tilde mit tiefgestelltem arg2
\DeclareRobustCommand{\Mt}[2]{\textbf{#1}_{#2}} %Matrix M mit tiefgestelltem arg2


\DeclareRobustCommand{\w}[1]{\underline{#1}} %Vektor arg1 unterstrichen 
\DeclareRobustCommand{\wt}[2]{\underline{#1}_{#2}} %Vektor unterstrichen w mit tiefgestelltem arg2
\DeclareRobustCommand{\wr}[1]{\underline{#1}_{\rho}} %Vektor unterstrichen mit rho, w_rho

% Inhalt
% ======
%	Makros für Referenzen (Abbildungen, Zitate, ...)
%	Makros für Abbildungen
%	Makros für Einheiten, Exponenten
%	Makros für Formeln
%	Makros für Entwurf
%   Definitionen für Umgebungen

% Makros für Abbildungen
% ======================
	% zum Skalieren nach Ersetzen durch psfrag
	\newcommand*{\incgraphicsw}[2]{\resizebox{#1}{!}{\includegraphics{#2}}}


% Textbausteine
% =============
	% Produktnamen
	\newcommand*{\Matlab}{\textsc{Matlab}}
	\newcommand*{\Matlabreg}{\textsc{Matlab}\textsuperscript{\tiny \textregistered}}
	\newcommand*{\MatSim}{\textsc{Matlab/Simulink}}
	\newcommand*{\Simulink}{\textsc{Simulink}}
	\newcommand*{\Simulinkreg}{\textsc{Simulink}\textsuperscript{\tiny \textregistered}}
	
	% Das Makro |\name|\marg{person} formatiert einen Personennamen bspw. eines Erfinders oder Entdeckers gemäß |\name{Euler}| \arrow\ \name{Euler}.
	\newcommand*{\name}[1]{\textsc{#1}}



% Makros für Einheiten, Exponenten
% ================================

	\newcommand*{\unit}[1]{\ensuremath{\mathrm{#1}}}
	
	% Wert mit Einheit (mit kleinem Leerzeichen dazwischen), aus Text- UND Math-Modus
	\newcommand*{\valunit}[2]{\ensuremath{#1\,\mrm{#2}}}


	% "°C", im Text- oder Mathe-Modus
	\newcommand*{\degC}{
		\ifmmode
			^\circ \mrm{C}%
		\else
			\textdegree C%
		\fi}

	\newcommand*{\degree}{
		\ifmmode
			^\circ%
		\else
			\textdegree%
		\fi}
	
	% Für Exponentenschreibweise ( Anwendung: 123\E{3} )
	\newcommand*{\E}[1]{\ensuremath{\cdot 10^{#1}}}
	
	\newcommand*{\eexp}[1]{\ensuremath{\mathrm{e}^{#1}}}
	\newcommand*{\iu}{\ensuremath{\mathrm{j}}}

	\newcommand*{\todots}{\ensuremath{,\,\hdots,\,}}


% Makros für Formeln
% ==================

	% Definition für Vektor und Matizen
    \newcommand*{\mat}[1]{{\ensuremath{\boldsymbol{\mathrm{#1}}}}}
    \newcommand*{\ma}[1]{{\ensuremath{\boldsymbol{\mathrm{#1}}}}}
    \newcommand*{\mas}[1]{\ensuremath{\boldsymbol{#1}}}
    \newcommand*{\ve}[1]{\ensuremath{\boldsymbol{#1}}}
    \newcommand*{\ves}[1]{\ensuremath{\boldsymbol{\mathrm{#1}}}}

	\newcommand*{\AP}{\ensuremath{\mathrm{AP}}}
	\newcommand*{\doti}{\ensuremath{(i)^\cdot}}
	
	\newcommand*{\inprod}[2]{\ensuremath{\langle #1,\,#2 \rangle}}
	
	\newcommand*{\ul}[1]{\underline{#1}}

	% gerades "d" (z.B. für Integral)
	\newcommand*{\ud}{\ensuremath{\mathrm{d}}}
	
	% normaler Text in Formeln
	\newcommand*{\tn}[1]{\textnormal{#1}}
	
	% nicht-kursive Schrift in Formeln
	\newcommand*{\mrm}[1]{\ensuremath{\mathrm{#1}}}
	
	% gerades "T" für Transponiert
	\newcommand*{\transp}{\ensuremath{\mathrm{T}}}
	
	% gerades "rg"
	\newcommand*{\rang}{\ensuremath{\operatorname{rg}}}

	% Für geklammerte Ausdrücke mit Index (Subscript)
	% (einmal mit kursiven Index, einmal mit geradem Index)
	\newcommand*{\grpsb}[2]{\ensuremath{\left(#1\right)_{#2}}}
	\newcommand*{\grprsb}[2]{\ensuremath{\left(#1\right)_{\mathrm{#2}}}}

	% Ableitungen und Integrale
		% "normale" Ableitung (mit geraden "d"s)
		\newcommand*{\normd}[2]{\ensuremath{\frac{\mathrm{d}#1}{\mathrm{d}#2}}}
		\newcommand*{\normdat}[3]{\ensuremath{\left.\frac{\mathrm{d} #1}{\mathrm{d} #2}\right|_{#3}}}
	
		% Materielle Ableitung
		\newcommand*{\matd}[2]{\ensuremath{\frac{\mathrm{D} #1}{\mathrm{D} #2}}}
		\newcommand*{\matdat}[3]{\ensuremath{\left.\frac{\mathrm{D} #1}{\mathrm{D} #2}\right|_{#3}}}
	
		% Partielle Ableitung
		\newcommand*{\partiald}[2]{\ensuremath{\frac{\partial #1}{\partial #2}}}
		\newcommand*{\partialdat}[3]{\ensuremath{\left.\frac{\partial #1}{\partial #2}\right|_{#3}}}
	
	
	% Transformationen
	\newcommand*{\FT}[1]{\ensuremath{\mathfrak{F}\left\{#1\right\}}}
	\newcommand*{\FTabs}[1]{\ensuremath{\left|\mathfrak{F}\left\{#1\right\}\right|}}
	\newcommand*{\IFT}[1]{\ensuremath{\mathfrak{F}^{-1}\left\{#1\right\}}}
	\newcommand*{\DFT}[1]{\ensuremath{\mathrm{DFT}\left\{#1\right\}}}
	\newcommand*{\DFTabs}[1]{\ensuremath{\left|\mathrm{DFT}\left\{#1\right\}\right|}}
	\newcommand*{\Laplace}[1]{\ensuremath{\mathfrak{L}\left(#1\right)}}
	\newcommand*{\InvLaplace}[1]{\ensuremath{\mathfrak{L^{-1}}\left(#1\right)}}
	\newcommand*{\invtrans}{\ensuremath{\quad\bullet\!\!-\!\!\!-\!\!\circ\quad}}
	\newcommand*{\trans}{\ensuremath{\quad\circ\!\!-\!\!\!-\!\!\bullet\quad}}


	\newcommand*{\mlfct}[1]{\texttt{#1}}
	\newcommand*{\mlvar}[1]{\texttt{#1}}


	% Manche textcomp-Zeichen funktionieren mit dem TU-Design nicht, diese können dann mit diesem
	% Befehl gesetzt werden.
	\newcommand*{\textcompstdfont}[1]{{\fontfamily{cmr} \fontseries{m} \fontshape{n} \selectfont #1}}
	


% =================================================================================
% Defintionen für Mathe-Umgebungen
% =================================================================================
	
\ifcsmacro{theorem}{}{
	\newtheorem{theorem}{Satz}
}
\ifcsmacro{lemma}{}{
	\newtheorem{lemma}[theorem]{Lemma}	% Selber Zähler wie theorem
}
\ifcsmacro{definition}{}{
	\newtheorem{definition}{Definition}
}
% =================================================================================


% =================================================================================
% Defintionen für Beispiel-Umgebung
% =================================================================================
\makeatletter
\ifx\c@chapter\undefined
	\newcounter{chapter}
\fi
\ifx\c@examplenumber\undefined
	\newcounter{examplenumber}[chapter]% Neuer Counter bspnummer nummeriert nach Kapitel
\fi
\makeatother
\def\theexamplenumber{\thechapter.\arabic{examplenumber}}

\ifcsmacro{example}{}{
	\newenvironment{example}[1][]
	{\vskip 3\parskip plus 1pt minus 1pt \refstepcounter{examplenumber}
	\vspace{.3cm} \begin{addmargin}[1cm]{0cm} \noindent \textbf{Beispiel \theexamplenumber}: \textbf{#1}\par}
	{\end{addmargin} \par \vspace{.3cm}}
}

% Alternative, einfachere Beispielumgebung:
% \newtheorem{example}{Beispiel}
% =================================================================================

% =================================================================================
% Definitionen für Tabellen
% =================================================================================
% Spaltenstil zum Ausrichten von Zahlen am Dezimaltrennzeichen
\newcolumntype{d}[1]{D{.}{,}{#1}}


% =================================================================================
% Definitionen für Listingsumgebung
% =================================================================================

\lstloadlanguages{Matlab}

\lstdefinestyle{Matlab_colored_smallfont}{
	language = Matlab,
	keywords = {break,case,catch,continue,class,else,elseif,end,enumeration,for,function,global,if,methods,otherwise,persistent,properties,return,switch,try,while},
	tabsize = 4,
	framesep = 3mm,
	frame=tb,
	classoffset = 0,	
	basicstyle = \footnotesize\ttfamily,
	keywordstyle = \bfseries\color[rgb]{0,0,1},
	commentstyle = \itshape\color[rgb]{0.133,0.545,0.133},
	stringstyle = \color[rgb]{0.627,0.126,0.941},
	extendedchars = true,
	breaklines = true,
	prebreak = \textrightarrow,
	postbreak = \textleftarrow,
	%escapeinside = {(*@}{@*)},
	%moredelim = [s][\itshape\color[rgb]{0.5,0.5,0.5}]{[.}{.]},
	numbers = left,
	numberstyle = \tiny,
	stepnumber = 5
}

\lstdefinestyle{Matlab_colored}{
	language = Matlab,
	keywords = {break,case,catch,continue,class,else,elseif,end,enumeration,for,function,global,if,methods,otherwise,persistent,properties,return,switch,try,while},
	tabsize = 4,
	framesep = 3mm,
	frame=tb,
	classoffset = 0,	
	basicstyle = \ttfamily,
	keywordstyle = \bfseries\color[rgb]{0,0,1},
	commentstyle = \itshape\color[rgb]{0.133,0.545,0.133},
	stringstyle = \color[rgb]{0.627,0.126,0.941},
	extendedchars = true,
	breaklines = true,
	prebreak = \textrightarrow,
	postbreak = \textleftarrow,
	%escapeinside = {(*@}{@*)},
	%moredelim = [s][\itshape\color[rgb]{0.5,0.5,0.5}]{[.}{.]},
	numbers = left,
	numberstyle = \tiny,
	stepnumber = 5
}


\lstdefinestyle{C_colored_smallfont}{
	language=C,
	tabsize = 4,
	framesep = 3mm,
	frame=tb,	
	classoffset = 0,	
	basicstyle = \footnotesize\ttfamily,
	keywordstyle = \bfseries\color[rgb]{0,0,1},
	commentstyle = \itshape\color[rgb]{0.133,0.545,0.133},
	stringstyle = \color[rgb]{0.627,0.126,0.941},
	extendedchars = true,
	breaklines = true,
	prebreak = \textrightarrow,
	postbreak = \textleftarrow,
	%escapeinside = {(*@}{@*)},
	%moredelim = [s][\itshape\color[rgb]{0.5,0.5,0.5}]{[.}{.]},
	numbers = left,
	numberstyle = \tiny,
	stepnumber = 5
}

\lstdefinestyle{C_colored}{
	language=C,
	tabsize = 4,
	framesep = 3mm,
	frame=tb,
	classoffset = 0,	
	basicstyle = \ttfamily,
	keywordstyle = \bfseries\color[rgb]{0,0,1},
	commentstyle = \itshape\color[rgb]{0.133,0.545,0.133},
	stringstyle = \color[rgb]{0.627,0.126,0.941},
	extendedchars = true,
	breaklines = true,
	prebreak = \textrightarrow,
	postbreak = \textleftarrow,
	%escapeinside = {(*@}{@*)},
	%moredelim = [s][\itshape\color[rgb]{0.5,0.5,0.5}]{[.}{.]},
	numbers = left,
	numberstyle = \tiny,
	stepnumber = 5
}

% Unterstützung von Umlauten in listings und tcblistings (tcolorbox Paket)
\lstset{
	inputencoding=utf8,
	extendedchars=true,
	literate=%
		{á}{{\'a}}1
		{é}{{\'e}}1
		{í}{{\'i}}1
		{ó}{{\'o}}1
		{ú}{{\'u}}1
		{Á}{{\'A}}1
		{É}{{\'E}}1
		{Í}{{\'I}}1
		{Ó}{{\'O}}1
		{Ú}{{\'U}}1
		{à}{{\`a}}1
		{è}{{\`e}}1
		{ì}{{\`i}}1
		{ò}{{\`o}}1
		{ù}{{\`u}}1
		{À}{{\`A}}1
		{È}{{\'E}}1
		{Ì}{{\`I}}1
		{Ò}{{\`O}}1
		{Ù}{{\`U}}1
		{ä}{{\"a}}1
		{ë}{{\"e}}1
		{ï}{{\"i}}1
		{ö}{{\"o}}1
		{ü}{{\"u}}1
		{Ä}{{\"A}}1
		{Ë}{{\"E}}1
		{Ï}{{\"I}}1
		{Ö}{{\"O}}1
		{Ü}{{\"U}}1
		{â}{{\^a}}1
		{ê}{{\^e}}1
		{î}{{\^i}}1
		{ô}{{\^o}}1
		{û}{{\^u}}1
		{Â}{{\^A}}1
		{Ê}{{\^E}}1
		{Î}{{\^I}}1
		{Ô}{{\^O}}1
		{Û}{{\^U}}1
		{œ}{{\oe}}1
		{Œ}{{\OE}}1
		{æ}{{\ae}}1
		{Æ}{{\AE}}1
		{ß}{{\ss}}2
		{ç}{{\c c}}1
		{Ç}{{\c C}}1
		{ø}{{\o}}1
		{å}{{\r a}}1
		{Å}{{\r A}}1
		{€}{{\EUR}}1
		{£}{{\pounds}}1
}

		% oft verwendete Befehle aus RTM Vorlage
% Hier eigene Befehle einf�gen			% Eigene Befehle
% verwendete Pakete einbinden
\addbibresource{./bib/literature.bib}% Bibliographie einbinden
\ifSADAuseglossaries
	\makeglossaries
	\loadglsentries{./Glossar/Verzeichnisse.tex}%Glossar einbinden
\fi
\usepackage{placeins}
\usepackage[stable]{footmisc}
\usepackage{mathtools}
\usepackage{mathptmx}
\usepackage[percent]{overpic} 
\usepackage{rotating}
\usepackage{todonotes}
\usepackage{ifthen}
\usepackage{subcaption}
\usepackage{multicol}
\usepackage{pdfpages}
\usepackage{tikz}
\usepackage{graphicx}
\usepackage{pgfplots}
\usepackage{subcaption}
\usepackage{transparent}
\usepackage{listings}
\usepackage{amsmath}
\usepackage{makecell}
\usetikzlibrary{external}
\tikzexternalize[prefix={cache/}]
\usetikzlibrary{shapes, arrows, patterns}
\usetikzlibrary{decorations.pathreplacing, decorations.pathmorphing}
\usetikzlibrary{shapes.geometric}
\makeatletter
\newcommand{\myunderbrace}[2]{\settowidth{\bracewidth}{$#1$}#1\hspace*{-1\bracewidth}\smash{\underbrace{\makebox{\phantom{$#1$}}}_{#2}}}
\newcommand\MyLBrace[2]{%
	\left.\rule{0pt}{#1}\right\}\text{#2}}
\renewcommand*\env@matrix[1][*\c@MaxMatrixCols c]{%
	\hskip -\arraycolsep
	\let\@ifnextchar\new@ifnextchar
	\array{#1}}
\pgfdeclarepatternformonly[\LineSpace]{my north east lines}{\pgfqpoint{-1pt}{-1pt}}{\pgfqpoint{\LineSpace}{\LineSpace}}{\pgfqpoint{\LineSpace}{\LineSpace}}%
{
	\pgfsetcolor{\tikz@pattern@color}
	\pgfsetlinewidth{0.4pt}
	\pgfpathmoveto{\pgfqpoint{0pt}{0pt}}
	\pgfpathlineto{\pgfqpoint{\LineSpace + 0.1pt}{\LineSpace + 0.1pt}}
	\pgfusepath{stroke}
}
\makeatother

\newdimen\LineSpace
\tikzset{
	line space/.code={\LineSpace=#1},
	line space=10pt
}

% =================================================================================
% Hier Daten für studentische Arbeit eingeben
% =================================================================================
%\newcommand*{\SADATyp}{Masterarbeit}
\newcommand*{\SADATitel}{Automatisierte Luftbetankung von Passagierflugzeugen}
%\newcommand*{\SADASubTitel}{Untertitel, der bei Abschlussarbeiten ignoriert wird}
%\newcommand*{\SADASubSubTitel}{Untertitel der ebenfalls ignoriert wird}
\newcommand*{\SADAStadt}{Darmstadt}
\newcommand*{\SADAAutor}{Markus Amann, Lukas Nevermann, Matheus Schelp Langendyk, Jinyuan Zhang} % Kommagetrennte Liste der Autoren
\newcommand*{\SADABetreuerI}{Philipp Schaub, M.Sc.}
%\newcommand*{\SADABetreuerII}{\DiplIng{} Hans Hilflos}
\newcommand*{\SADABetreuerIII}{}
\newcommand*{\SADABegin}{01. November 2020}
\newcommand*{\SADAAbgabe}{15. März 2021}
\newcommand*{\SADASeminar}{09. April 2021}


\usepackage[
  % TUDmargin=true,    % Wenn ein Breiter Rand für Korrekturen gewünscht ist
	onlycolorfront=true, % Wenn nur die erste Seite einen farbigen Balken erhalten soll
	fachgebiet=RTM,      % RTM / RTP
	fachbereich=18,      % 18 / 16
	typ=PS,              % MA=Masterarbeit  BA=Bachelorarbeit PS=Projektseminar PR=Proseminar SA=Studienarbeit DA=Diplomarbeit  
	                     % script=Skript exercise=Übung report=Bericht internship=Praktikumsbericht  user=Eigener Typ, der mit \newcommand*{\SADATyp}{MeinTyp} gesetzt werden kann
	accentcolor=tud2c,   % Wird bei der automatischen Farbumschaltung verwendet.
	tikzexternal=none	 % none=keine Externalisierung, make=IAT Externalisierung, tikz=tikzexternalize
]{iatsada}

\ifcsname setcooperationlogo\endcsname
	\setcooperationlogo[height]{./Bilder/Firmenlogo}% Bei Bedarf kann ein Firmenlogo gesetzt werden, wenn tudthesis entsprechend gepatcht wurde
\fi
\setkeysinputtikz{path=./Bilder/, errorlevel=warning}
\iatsadatikzexternalmode{tikz}{
	\tikzexternalize[shell escape=-enable-write18]%tikzexternal aktivieren
}{}


% =================================================================================
% Texte für die Rückseite des Titelblatts vorgeben
% =================================================================================
\uppertitleback{}
\lowertitleback{}
\dedication{}
% =================================================================================
% Hier beginnt das eigentliche Dokument
% =================================================================================


% =================================================================================
% kann nach Ersetzung der Hilfetexte gelöscht werden
% =================================================================================
\usepackage{forest}
\definecolor{folderbg}{RGB}{124,166,198}
\tikzset{
	folder/.pic={
		\filldraw[draw={rgb:red,110;green,144;blue,169},top color=folderbg!50,bottom color=folderbg]
		(-4.2pt,5.8pt) rectangle ++(3pt,-5.8pt);
		\filldraw[draw={rgb:red,110;green,144;blue,169},top color=folderbg!50,bottom color=folderbg]
		(-4.6pt,-4pt) rectangle (4.6pt,4pt);
	}
}
\newcommand*{\Miktex}{\textsc{MiKTeX}\xspace}
\newcommand*{\texlive}{\textsc{texlive}\xspace}
\newcommand{\Texstudio}{\textsc{TeXStudio}\xspace}
\newcommand{\Texniccenter}{\textsc{TeXnicCenter}\xspace}
\newcommand{\Sumatra}{\textsc{Sumatra}\xspace}
\newcommand*{\zitat}[1]{\glqq{}#1\grqq{}}