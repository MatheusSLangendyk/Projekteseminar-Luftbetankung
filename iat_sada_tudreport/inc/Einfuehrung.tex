\chapter{Struktur}\label{cha:struktur}
\section{Variablenverzeichnis}
- Aufpassen bei vergabe von variablennamen --> nicht mehrfach verwenden
\section{Abkürzungsverzeichnis}
\section{Aufgabenstellung}
\section{Einleitung} 
--> Lorcas\\
- 1-2 Seiten\\
\section{State of the Art und Allgemeines} 
- 5 Seiten\\
--> Jacky\\
- Trajektorien im Betankungsvorgang\\
- flugzeuglage relativ zueinander erklären (Abstand, winkel und länge des tankschlauchs)\\
- tanker fliegt hinten und unten damit er besser auf störungen reagieren kann\\
\section{Parameter, Notation, Modellbildung, Arbeitspunkt, Linearisierung, Zweiflugzeug-Modell, Störmodelle (Masse, evtl. Wind)}
- 10-15 Seiten\\
--> Matheus\\
- koordinatensysteme\\
- AP erklären\\
- bezeichnung nominelles system wegen masse\\
- bezeichnung flugzeug 1 und 2 \\
- erklären, welches Flugzeug führungsflugzeug ist\\
- Wahl der zustände und Ausgänge\\
- massenmodell erklären\\
- kurz Windmodell erklären (mit Referenz)\\
- Definition von allg systemparametern wie A,B,C,n,p,q etc.\\
- Dabei dann auch erklären warum die sehr kleinen Werte nicht entfernt wurden \\
\section{Anforderungen an die Regelung}
--> Markus\\
- 2 Seiten\\
\section{Theorie der verwendeten Regelungsverfahren}
--> Lorcas\\
- 5 Seiten\\
\subsection{Polbereichsplatzierung}
\subsection{Robuste Verkopplungsregelung mit geregelt invariantem Unterraum}
- allgemeine Erklärung zu Verkopplungsregelung\\
- Erklärung der berechnung des modifizierten Vorfilters
- Theorie zu geregelt invariantem Unterraum \\
- Erklärung was das für die Regelgrößen bedeutet (nur w1 nutzbar)\\
\subsection{gammasyn}
- gammasyn --> warum? vorteile (gleichzeitig entkopplung, verkopplung, polbereichsplatzierung)\\
- Herleitung der tf\_structure (Verkoopplung deswegen SW Teil gleich null, Eingänge w2 nicht benutzt deswegen O Teil gleich NaN, Entkopplung bei F1 deswegen NW Teil NaN und 0 Struktur)\\

\section{Entwurf eines robusten Verkopplungsreglers für das nominelle Zwei-Flugzeug-Modell (gleiche Masse)}
--> Markus\\
- 10-12 Seiten \\
- Eigenwerte des Systems\\
- Wahl des Polgebiets\\
- Welcher REgler kommt raus?\\
- Wie sehen die Ergebnisse aus?\\
- Plots pzmap und step von nominellem Modell (linear)\\
- Ergebnisse von nominellem Modell (nichtlinear) --> Sprünge Sollwerte, Stellgrößen, Verkopplung der Ausgänge, Verkopplung der Positionen\\
- Robustheitsanalyse Parameterschwankung --> pzmap und step bei Änderung der Masse --> Regler ist nicht robust\\
- Demonstration der "Nicht-Robustheit" am nichtlinearen Modell (was passiert, wen man die Modellmasse ändert)\\
- Strukturelle Erweiterungen (PID-Regelung) für nominelles Modell zur Regelung von sprungförmigen Störungen für stationäre Genauigkeit \\
--> Ergebnisse der PID-Regelung bei Wind, Zustandsänderungen (Luftloch) und Massenfluss\\
\section{Zusammenfassung und Ausblick}
--> Lorcas\\
- 1-2 Seiten\\
\subsection{Fazit}
- Verkopplung für lineares und nichtlineares Modell erfolgreich \\
- sobald Stellgrößenbeschr. dann wird System bei Anregung von $\Theta$ instabil \\
- ansonsten stationär genau, in den anderen Fällen stellgrößenbeschr eingehalten etc... \\
- in den anderen Fällen (also alles was nicht $\Theta$ ist) ist im Fall für Führungsverhalten, also keine Störung, auch die Positionsverkopplung gegeben  \\
- Multi-Modell-Ansatz failt auch schon für extrem kleine Schwankungen (\valunit{1.167}{\%}) --> Verkopplung nicht mehr möglich, system wird instabil wegen EW in rechter s-halbebene, also nicht robust \\
- zeigt sich auch wenn man die Modelle mit Massenschwankung mit unseren Regler regelt, der für das nominelle system ohne multi-modell-ansatz entworfen wurde (dann bleibt halt ne abweichung, verkopplung nicht erfüllt, auch die positionen weichen ab)\\
--> keine robustheit gegnüber parameterschwankung in masse\\

- mit unterlagerung können sprungförmige störungen ausgeregelt werden \\
- bei luftloch bleibt verkopplung erhalten auch in den positionen \\
- bei wind bleibt verkopplung in den ausgängen erhalten, aber nicht in den positionen (flugzeuge entfernen sich in y-richtung)\\
--> robustheit gegenüber störeingängen, die die zustände direkt anregen \\
--> aber es zeigt sich, dass die positionsverkopplung nur über euler winkel nicht erreicht werden kann

\subsection{Ausblick}
- Unterschieldiche Flugzeuge modellieren --> andere Dynamik\\
- Erweiterung der Regelung um eine Positionsregelung (Verkopplung von x,y)\\
- Modellierung der Sensorik\\
- Verbesserung der Robustheit gegnüber massenschwankung (Abhängigkeit vom AP, Abhängigkeit von Masse)\\
- evtl(\textit{Verbesserung der Robustheit --> krasse Abhängigkeit von Theta --> Verkopplung ohne theta})\\
- evtl(\textit{warum ist gammasyn so kacke})
\section{Anhang}
\subsection{Simulink Modell}
\subsection{Visualisierung}
- innerhalb von Matlab/ Simulink\\
- Flight Gear \\
\section{Literaturverzeichnis}
- was referenzieren wir für allgemeines? Föllinger oder Mehrgrößenreglerentwurf?
%Im Anschluss daran finden sich in zwei Kapiteln Hinweise zur Erstellung der schriftlichen Arbeit. Kapitel~\ref{cha:Hinweise-Allgemein} gibt allgemeine Hinweise, wohingegen in Kapitel~\ref{cha:Hinweise-Latex} zusätzlich \LaTeX-spezifische Befehle vorgestellt werden. Je nach verwendetem Programm, kann das eine oder andere Kapitel übersprungen werden.

