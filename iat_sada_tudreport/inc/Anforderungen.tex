\chapter{Anforderungen an die Regelung}\label{cha:Anforderungen}
Dieses Kapitel dient der Herleitung und Erklärung von Anforderungen, die an die Regelung gestellt werden. Diese ergeben sich aus Aspekten der 
Sicherheit, des Komforts und der physikalischen Realisierbarkeit. In den darauf folgenden Kapiteln wird ein Verkopplungsregler entworfen und die damit erzielten Ergebnisse diskutiert. Dabei spielt die Validierung des geregelten Modells hinsichtlich der in diesem Kapitel vorgestellten Anforderungen eine zentrale Rolle.

\section{Stabilität, Dämpfung, stationäre Genauigkeit}
Die wohl trivialste und auch wichtigste Anforderung ist die nach Stabilität. Selbstverständlich sollte der Regler dazu in der Lage sein, das System zu stabilisieren. Dies gilt sowohl für den Arbeitspunkt als auch für geringe Änderungen um diesen. Des Weiteren sollte das dynamische Verhalten der Flugzeuge möglichst gut gedämpft sein, um stärkere Schwingungen zu vermeiden, da diese den Komfort stark einschränken würden. Zusätzlich soll die Regelung stationär genau arbeiten. Dies ist vor allem deshalb wichtig, da die Flugzeuge während der Betankung sehr nah bei einander fliegen und den Sollwertvorgaben daher exakt und stationär genau nachgekommen werden muss. Außerdem darf die Entfernung der Flugzeuge die Reichweite des Auslegers nicht überschreiten.

\section{Stellgrößenbeschränkungen}
Die Einhaltung der maximalen bzw. minimalen Stellgrößen ist ein essenzieller Aspekt bei der Bewertung des Verhaltens der Regelstrecke. Dies gilt sowohl für das im Arbeitspunkt linearisierte Modell als auch für das nichtlineare Modell. Die Stellgrößenbeschränkungen stellen Grenzwerte für die Ruder und den Antrieb des Flugzeugs dar, welche dem Einfluss, der über die Stellgrößen auf das Flugzeug genommen werden kann, einen realistischen Rahmen geben. Auf diese Weise wird verhindert, dass unrealistische Werte (bspw. Ruderwinkel von über $180\degree$, oder Antriebswerte jenseits der maximal verfügbaren Schubkraft) auf das Modell geschaltet werden. Gleichzeitig sollte bei der Auslegung des Reglers darauf geachtet werden, dass sich die benötigten Stellgrößen ohnehin innerhalb dieser Grenzen bewegen, da das System sonst unerwünschtes Verhalten aufweisen kann. 

\section{Verkopplung der Flugzeuge}
Eine maßgebliche Aufgabe dieser Arbeit besteht darin, die Verkoppelbarkeit der beiden Flugzeuge zu untersuchen. Folglich besteht eine Anforderung, die die Regelung erfüllen sollte, in der Verkopplung der Ausgangsgrößen der Flugzeuge. Unter der Annahme, dass sich die Flugzeuge zu Beginn der Regelung in der jeweiligen Sollposition, d.h. am jeweiligen Arbeitspunkt, befinden, soll über die Verkopplung der Ausgänge schließlich die Verkopplung der Flugzeugpositionen bezüglich des erdfesten KS erreicht werden. 

\section{Teilentkopplung der Ausgänge des Führungsflugzeugs} 
Neben der Forderung nach Verkopplung der Ausgänge der beiden Flugzeuge, soll zudem die teilweise Entkopplung der Ausgänge von Flugzeug 1 erreicht werden. Grund dafür ist, dass bei Systemen mit mehreren Ein- und Ausgangsgrößen (MIMO-Systeme) in der Regel eine innere Verkopplung zwischen den einzelnen Ein- und Ausgängen besteht. Dies führt dazu, dass sich Sollwertvorgaben für einzelne Eingangsgrößen im Allgemeinen auf alle Ausgänge auswirken, wodurch MIMO-Systeme deutlich schwieriger zu regeln sind als SISO-Systeme. Um dies zu vermeiden, wird zusätzlich die teilweise Entkopplung der Ausgänge von Flugzeug 1 gefordert. Auf die genaue Wunschstruktur der Übertragungsmatrix des Gesamtsystems wird in Kapitel \ref{cha:Regler} eingegangen.

\section{Robustheit}
Des Weiteren soll die Regelung möglichst robust sein. Dies betrifft Abweichungen vom Arbeitspunkt sowie Einflüsse durch Parameterschwankungen und äußere Störungen, die im Folgenden erklärt werden.

\subsection{Parameterschwankungen durch variable Flugzeugmasse}
In dem hier verwendeten Modell, welches in Kapitel \ref{cha:Modellbildung} hergeleitet wurde, wird die zeitlich konstante Flugzeugmasse $m$ angenommen. Während des Betankungsvorgangs, in dem die Regelung aktiv ist, wird jedoch Masse zwischen den beiden Flugzeugen ausgetauscht. Der Massenfluss $\dot{m}_\mrm{ts}$, der zwischen den beiden Flugzeugen ausgetauscht wird, führt dazu, dass die Masse des Tankflugzeugs abnimmt und die des zu betankenden Flugzeugs zunimmt. Ferner muss die Tankladung vom Tankflugzeug ohnehin erst in die Luft befördert werden, sodass die Gesamtmasse des Tanker schon zu Beginn des Vorgangs von der Flugzeugmasse $m$ abweicht. Daher ist Robustheit der Regelung hinsichtlich dieser Parameterschwankung wünschenswert.

\subsection{Äußere Störeinflüsse}
Auch äußere Störeinflüsse, die in Form von Wind oder Zustandsänderungen (bspw. Luftloch) auf die Flugzeuge einwirken, beeinflussen das dynamische Verhalten. Trotz dieser Einflüsse sollte die Regelung in der Lage sein, die Flugzeuge zu stabilisieren und miteinander zu verkoppeln, da sonst bei Windböen die Gefahr eines Absturzes besteht. Insbesondere die Nähe, die die Flugzeuge während der Betankung zu einander haben, führt zu der Forderung, dass derartige Störeinflüsse möglichst gut und schnell ausgeregelt werden sollten. 