\section*{Aufgabenstellung}

Nach einer Studie der Zürcher Hochschule für Angewandte Wissenschaften (ZHAW) aus dem Jahr 2015 \footnote{ZHAW, „Fliegende Tankstellen könnten den Luftverkehr revolutionieren“, https://www.zhaw.ch/de/medien/
	medienmitteilungen/detailansicht-medienmitteilung/event-news/fliegende-tankstellen-koennten-den-luftverkehr-revolutionieren/, abgerufen 5.10.2020} können durch Luftbetankung im zivilen Luftverkehr erhebliche Mengen an Treibstoff eingespart werden.

Um den Zusatzaufwand für Piloten möglichst gering zu halten, soll in diesem Projektseminar ein Konzept zur automatisierten Luftbetankung entwickelt werden. Dazu zählen die Herleitung eines geeigneten nicht-linearen Systemmodells inklusive der Modellierung von Störungen (z.B. Wind, Delays durch Kommunikation), die Systemvereinfachung, Linearisierung um relevante Arbeitspunkte sowie der Beobachter- und Verkopplungseglerentwurf per Zustands- und Ausgangsrückführung. Dabei sind die mindestens notwendigen Messgrößen zu ermitteln.

Die Regelungen sollen simulativ (\textsc{Matlab-Simulink}) evaluiert werden und auf Ihre Robustheit bezüglich Störungen und Parameterunsicherheiten der Regelstrecken untersucht werden. Sie sollen außerdem mit robusten Verkopplungsregelungen verglichen werden. Für deren Entwurf steht die instituseigene Toolbox \texttt{gammasyn} zur Verfügung.

Es ist zu erforschen, in welchem Rahmen Flugmanöver (z.B. Kurswechsel, Höhenänderung, Beschleunigung) während der Betankung durchgeführt werden können.

Die Simulationsergebnisse sind visuell darzustellen. Dazu können bestehende Animierungsumgebungen zur Anwendung kommen.

Sämtliche Ergebnisse sind ausführlich zu dokumentieren. Die aktuelle Fassung der Richtlinien zur Anfertigung von Abschlussarbeiten ist zu beachten.


\SADAAufgabenstellung