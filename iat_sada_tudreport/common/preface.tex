\selectMainLanguage
\SADAmaketitle
\pagenumbering{roman}	% Bis zum Hauptteil werden römische Seitenzahlen verwendet
\SADAfirstpage

\ifSADAfinal% Aufgabestellung nur in der Abgabeversion einbinden
	% =================================================================================
	% Spezielle Seiten für studentische Arbeiten
	% =================================================================================
	\cleardoublepage
	\section*{Aufgabenstellung}

Nach einer Studie der Zürcher Hochschule für Angewandte Wissenschaften (ZHAW) aus dem Jahr 2015 \footnote{ZHAW, „Fliegende Tankstellen könnten den Luftverkehr revolutionieren“, https://www.zhaw.ch/de/medien/
	medienmitteilungen/detailansicht-medienmitteilung/event-news/fliegende-tankstellen-koennten-den-luftverkehr-revolutionieren/, abgerufen 5.10.2020} können durch Luftbetankung im zivilen Luftverkehr erhebliche Mengen an Treibstoff eingespart werden.

Um den Zusatzaufwand für Piloten möglichst gering zu halten, soll in diesem Projektseminar ein Konzept zur automatisierten Luftbetankung entwickelt werden. Dazu zählen die Herleitung eines geeigneten nicht-linearen Systemmodells inklusive der Modellierung von Störungen (z.B. Wind, Delays durch Kommunikation), die Systemvereinfachung, Linearisierung um relevante Arbeitspunkte sowie der Beobachter- und Verkopplungseglerentwurf per Zustands- und Ausgangsrückführung. Dabei sind die mindestens notwendigen Messgrößen zu ermitteln.

Die Regelungen sollen simulativ (\textsc{Matlab-Simulink}) evaluiert werden und auf Ihre Robustheit bezüglich Störungen und Parameterunsicherheiten der Regelstrecken untersucht werden. Sie sollen außerdem mit robusten Verkopplungsregelungen verglichen werden. Für deren Entwurf steht die instituseigene Toolbox \texttt{gammasyn} zur Verfügung.

Es ist zu erforschen, in welchem Rahmen Flugmanöver (z.B. Kurswechsel, Höhenänderung, Beschleunigung) während der Betankung durchgeführt werden können.

Die Simulationsergebnisse sind visuell darzustellen. Dazu können bestehende Animierungsumgebungen zur Anwendung kommen.

Sämtliche Ergebnisse sind ausführlich zu dokumentieren. Die aktuelle Fassung der Richtlinien zur Anfertigung von Abschlussarbeiten ist zu beachten.


\SADAAufgabenstellung

	\cleardoublepage
	\selectMainLanguage
	
	\SADAErklaerung

%	\clearpage
%	\selectlanguage{ngerman}
\section*{Kurzfassung}
Das \LaTeX-Dokument \verb|sada_tudreport| ist eine Vorlage für schriftliche Arbeiten (Proseminar-, Projektseminar-, Studien-, Bachelor-, Master- und Diplomarbeiten, \etc) am Institut für Automatisierungstechnik der TU Darmstadt. Das
Layout ist an die \emph{Richtlinien zur Anfertigung von Studien- und
Diplomarbeiten}~\cite{Richtlinien} angepasst und durch Modifikation der Klasse \verb|tudreport|
realisiert, so dass in der Arbeit die gewohnten \LaTeX-Befehle benutzt werden
können. Die vorliegende Anleitung beschreibt die Klasse und gibt grundlegende
Hinweise zum Verfassen wissenschaftlicher Arbeiten. Sie ist außerdem ein
Beispiel für den Aufbau einer Studien-, Bachelor-, Master- bzw. Diplomarbeit.

\SADAkeywords{Studienarbeit, Bachelorarbeit, Masterarbeit, Diplomarbeit, Vorlage, \LaTeX-Klasse}
\selectlanguage{ngerman}



\selectlanguage{english}
\section*{Abstract}
The \LaTeX\ document \verb|sada_tudreport| provides a template for student's research
reports and diploma theses (`` Proseminar-, Projektseminar-, Studien-, Bachelor-, Master- und Diplomarbeiten'') at the Institute of
Automatic Control, Technische Universität Darmstadt. The layout is adapted to
the \emph{``Richtlinien zur Anfertigung von Studien- und
Diplomarbeiten''}~\cite{Richtlinien} and is implemented by modification of the standard \verb|tudreport|
class, so that common \LaTeX\ commands can be used in the text. This manual
describes the class and dwells on general considerations on how to write
scientific reports. Additionally, it is an example for the structure of a
thesis.

\SADAkeywords{Research reports, diploma theses, template, \LaTeX\ class}
\selectlanguage{ngerman} 
	% =================================================================================
\fi

\selectMainLanguage

% =================================================================================
% Inhaltsverzeichnis
% =================================================================================
\cleardoublepage	% Auf einer leeren rechten Seite beginnen
\phantomsection		% Diese und die nächste Zeile dient dazu, für das Inhalts-
					% verzeichnis einen Eintrag in das pdf-Inhaltsverzeichnis,
					% aber nicht in das normale Verzeichnis zu erzeugen.
\pdfbookmark[0]{\contentsname}{pdf:toc}
\tableofcontents	% Inhaltsverzeichnis einfügen
\clearpage			% Sonst kommt nichts mehr auf die Seite
% =================================================================================

%\ifSADAfinal% Verzeichnisse nur in der Abgabeversion einbinden
%	% =================================================================================
%	% Symbole und Abkürzungen
%	% =================================================================================
%	% Nach dem Inhaltsverzeichnis kommt ein Verzeichnis der häufig verwendeten Symbole und Abkürzungen.
%	\cleardoublepage
%	\ifSADAuseglossaries
%		\glsaddallunused[\acronymtype,symbols,main]%alle definierten Einträge im Symbolverzeichnis anzeigen
%		\glssetwidest[1]{\hspace{3cm}}%breitestes Symbol im Symbolverzeichnis
%		\printacronyms[style=long]
%	\else
%		\chapter*{Abkürzungsverzeichnis}
\phantomsection
\addcontentsline{toc}{chapter}{Abkürzungsverzeichnis}% erzeugt einen Eintrag im Inhaltsverzeichnis
\begin{tabularx}{\textwidth}{@{}l@{\qquad}X}
	Kürzel	&	vollständige Bezeichnung\\\midrule
	Dgl.	&	Differentialgleichung\\
	LS		&	Kleinste Quadrate (\emph{Least Squares})\\
	PRBS	&	Pseudo-Rausch-Binär-Signal (\emph{Pseudo Random Binary Signal})\\
	ZVF		&	Zustandsvariablenfilter
\end{tabularx}
%	\fi
%	\cleardoublepage
%	\ifSADAuseglossaries
%		\newglossarystyle{symbollist}{
%			\setglossarystyle{alttreegroup}
%			\renewcommand{\glsgroupskip}{}
%			\renewcommand{\glsgetgrouptitle}[1]{}
%		}
%		\printsymbols[style=symbollist,nonumberlist]
%	\else
%		\chapter*{Symbolverzeichnis}
\phantomsection
\addcontentsline{toc}{chapter}{Symbolverzeichnis}% erzeugt einen Eintrag im Inhaltsverzeichnis
%
\paragraph*{Lateinische Symbole und Formelzeichen}
\begin{tabularx}{\textwidth}{@{}l@{\qquad}X@{\quad}p{18mm}}
	Symbol	&	Beschreibung	&	Einheit\\\midrule
	$I$		&	Strom			&	\unit{A}\\
	$R$		&	Widerstand		&	\unit{\Omega}\\
	$U$		&	Spannung		&	\unit{V}\\
\end{tabularx}
%
\paragraph*{Griechische Symbole und Formelzeichen}
\begin{tabularx}{\textwidth}{@{}l@{\qquad}X@{\quad}p{18mm}}
	Symbol			&	Beschreibung		& Einheit\\\midrule
	$\mat{\Psi}$	&	Datenmatrix			&	\\
	$\sigma$		&	Standardabweichung	&	\\
	$\omega$		&	Kreisfrequenz		&	\unit{s^{-1}}
\end{tabularx}
%	\fi
%	\cleardoublepage
%\fi
% =================================================================================
% Hauptteil
% =================================================================================
\pagenumbering{arabic}	% Hauptteil bekommt arabische Seitenzahlen