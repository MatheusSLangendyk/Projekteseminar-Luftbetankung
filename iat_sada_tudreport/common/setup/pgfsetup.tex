\usepackage{fp} 
\usepackage{tikz}				% Zum Erzeugen von Bildern mit TikZ
\usepackage{pgf}
\usepackage{pgfplots}			% Zum Erzeugen von Diagrammen mit pgfplots
\usepackage{pgfplotstable}
\usetikzlibrary{arrows}
\usetikzlibrary{arrows.meta}
\usetikzlibrary{backgrounds}
\usetikzlibrary{calc}
\usetikzlibrary{circuits}
\usetikzlibrary{circuits.ee.IEC}
\usetikzlibrary{decorations.markings}
\usetikzlibrary{decorations.shapes}
\usetikzlibrary{decorations.text}
\usetikzlibrary{fadings}
\usetikzlibrary{fit}
\usetikzlibrary{intersections}
\usetikzlibrary{matrix}
\usetikzlibrary{patterns}
\usetikzlibrary{positioning}
\usetikzlibrary{shapes}
\usetikzlibrary{shadows}
\usetikzlibrary{spy}
\IfLanguageName{german}{
	\tikzset{
		/pgf/number format/use comma,
		/pgf/number format/1000 sep=\,
	}
}{}
\IfLanguageName{ngerman}{
	\tikzset{
		/pgf/number format/use comma,
		/pgf/number format/1000 sep=\,
	}
}{}
\usepgfplotslibrary{groupplots}
%\usepgfplotslibrary{fillbetween}

\pgfplotsset{compat=newest}
\IfLanguageName{german}{
	\pgfplotsset{
		x tick label style={/pgf/number format/use comma, /pgf/number format/1000 sep=\,},
		y tick label style={/pgf/number format/use comma, /pgf/number format/1000 sep=\,},
		z tick label style={/pgf/number format/use comma, /pgf/number format/1000 sep=\,}
	}
}{}
\IfLanguageName{ngerman}{
	\pgfplotsset{
		x tick label style={/pgf/number format/use comma, /pgf/number format/1000 sep=\,},
		y tick label style={/pgf/number format/use comma, /pgf/number format/1000 sep=\,},
		z tick label style={/pgf/number format/use comma, /pgf/number format/1000 sep=\,}
	}
}{}

\edef\pgfdatafolder{./Bilder/Daten} % Verzeichnis in dem die csv Dateien für pgfplots liegen

\newcommand{\xmin}{1e-2}
\newcommand{\xmax}{1e2}
\newcommand{\mywidth}{0.8\textwidth}
\newcommand{\myheight}{60mm}

\newcommand{\omegaD}{1}

\newcommand{\bodestyle}{
	\pgfplotsset{
		major grid style={line width=0.3pt, color=gray},
		minor grid style={line width=0.3pt, color=gray},
		major tick style={line width=0.4pt, color=black},
		major tick length={4pt},
		minor tick length={3pt},
		tick label style={font=\small}
	}
}

\newcommand{\nyquiststyle}{
	\pgfplotsset{
		major grid style={line width=0.3pt, color=gray},
		minor grid style={line width=0.3pt, color=gray},
		major tick style={line width=0.4pt, color=black},
		major tick length={4pt},
		minor tick length={3pt},
		tick label style={font=\small}
	}
}

\newcommand{\plotstyle}{
	\pgfplotsset{
		major grid style={line width=0.3pt, color=gray},
		minor grid style={line width=0.3pt, color=gray},
		major tick style={line width=0.4pt, color=black},
		major tick length={4pt},
		minor tick length={3pt},
		tick label style={font=\small}
	}
}

\newcommand{\plotyystyle}{
	\pgfplotsset{
		every non boxed y axis/.style={ytick align=inside,y axis line style={-}},
		every boxed y axis/.style={}
	}
}

\newenvironment{bodeAmpDB}[1][]{
\bodestyle
\begin{semilogxaxis}[
	ylabel=$|G(\mathrm{j}\omega)|_{\mathrm{dB}}$,
	%ylabel=Amplitude in dB,
	ylabel style={yshift=2pt},
	enlarge x limits=false,
	xminorgrids=true,
	xmajorgrids=true,
	ymajorgrids=true,
	yminorgrids=true,
	xticklabels=\empty,
	width=\mywidth,
	height=\myheight,#1]
}{\end{semilogxaxis}}

\newenvironment{bodeAmpLOG}[1][]{
\begin{loglogaxis}[
	ylabel=$|G(\mathrm{j}\omega)|$,
	%ylabel=Amplitude,
	ylabel style={yshift=2pt},
	enlarge x limits=false,
	xminorgrids=true,
	xmajorgrids=true,
	ymajorgrids=true,
	yminorgrids=true,
	xticklabels=\empty,
	y tick label style={font=\small},
	width=\mywidth,
	height=\myheight,#1]
}{\end{loglogaxis}}

\newenvironment{bodePhase}[1][]{
\begin{scope}[yshift=-\myheight+12mm]	% zweiten Plot (Phase) unter den Amplitudengang setzen
	\bodestyle
	\begin{semilogxaxis}[
		xlabel=Frequenz $\omega$ in \ensuremath{\mathrm{{\frac{rad}{s}}}},
		%ylabel=Phase in $^\circ$,
		ylabel=$\angle{G(\mathrm{j}\omega)}$,
		ylabel style={yshift=2pt},
		enlarge x limits=false,
		xminorgrids=true,
		xmajorgrids=true,
		ymajorgrids=true,
		yminorgrids=true,
		ytick={-360, -315,..., 360},
		yticklabel={$\pgfmathprintnumber{\tick}^\circ$},% ° als Einheitenzeichen an alle yticks
		width=\mywidth,
		height=\myheight,#1]
}{\end{semilogxaxis}\end{scope}}

\newenvironment{plotyyLeft}[1][]{
\plotyystyle
\begin{axis}[
	scale only axis,
	enlarge x limits=0,
	axis y line=left,
	width=\mywidth,
	height=\myheight,#1]
}{\end{axis}}

\newenvironment{plotyyRight}[1][]{
\plotyystyle
\begin{axis}[
	scale only axis,
	enlarge x limits=0,
	axis y line=right,
	axis x line=none,
	width=\mywidth,
	height=\myheight,#1]
}{\end{axis}}

\newcommand{\TikZpole}[1][blue]{
\begin{tikzpicture}
	\draw[#1, very thick, line cap=round] (-1,1)  -- (1,-1);
	\draw[#1, very thick, line cap=round] (-1,-1) -- (1,1);
\end{tikzpicture}
}

\newcommand{\TikZzero}[1][blue]{
\begin{tikzpicture}
	\draw[#1, very thick] (0,0) circle (5pt);
\end{tikzpicture}
}

% Blockschaltbilder
\newcommand{\TikZscale}{1}
\newcommand{\mm}{*\TikZscale mm}
% TikZstyles f�r Blockschaltbilder

\renewcommand{\TikZscale}{1}
\tikzset{every picture/.style={node distance=4\mm, >=stealth'}}

%% Bl�cke
	% Rechteckige Bl�cke
	\tikzstyle{block}      = [draw, semithick, rectangle, minimum height=8\mm, minimum width=8\mm, inner sep=3pt]
	\tikzstyle{NLblock}    = [draw, semithick, rectangle, minimum height=8\mm, minimum width=8\mm, inner sep=3pt, double distance=1.2pt]
	\tikzstyle{PICblock}   = [draw, semithick, rectangle, minimum height=8\mm, minimum width=8\mm, inner sep=2pt]
	\tikzstyle{NLPICblock} = [draw, semithick, rectangle, minimum height=8\mm, minimum width=8\mm, inner sep=3pt, double distance=1.2pt]
	\tikzstyle{noblock}	   = [rectangle, inner sep=-0.6pt]

	% Dreieckige Bl�cke
	\tikzstyle{Rgain}			 = [draw, semithick, isosceles triangle, inner sep=1pt, minimum height=8\mm, isosceles triangle apex angle=60]
	\tikzstyle{Lgain}			 = [draw, semithick, isosceles triangle, inner sep=1pt, minimum height=8\mm, isosceles triangle apex angle=60, shape border rotate=180]
	\tikzstyle{Ugain}			 = [draw, semithick, isosceles triangle, inner sep=1pt, minimum height=8\mm, isosceles triangle apex angle=60, shape border rotate=90]
	\tikzstyle{Dgain}			 = [draw, semithick, isosceles triangle, inner sep=1pt, minimum height=8\mm, isosceles triangle apex angle=60, shape border rotate=-90]

	% Runde Bl�cke
	\tikzstyle{sum}   	   = [draw, semithick, circle, inner sep=1pt, minimum size=3\mm]
	\tikzstyle{branch}		 = [draw, circle, inner sep=0pt, minimum size=1\mm, fill=black]
	\tikzstyle{BRANCH}		 = [coordinate]


%% Verbindungselemente
	% Linien mit Pfeil
	\tikzstyle{to}  		= [->, thick]
	\tikzstyle{toNL}		= [->, thick, shorten >=0.9pt]
	\tikzstyle{NLto}		= [->, thick, shorten <=0.9pt]
	\tikzstyle{NLtoNL}	= [->, thick, shorten <=0.9pt, shorten >=0.9pt]
	
	\tikzstyle{TO}  		= [semithick, double distance=2pt, shorten >=2mm, decoration={markings,mark=at position 1 with {\arrow[semithick]{open triangle 60}}}, preaction={decorate},postaction={draw, line width=2pt, white, shorten >= 1.5mm}]
	
	\tikzstyle{TONL}  		= [semithick, double distance=2pt, shorten >=2.2mm, decoration={markings,mark=at position 1 with {\arrow[semithick]{open triangle 60}}, transform={xshift=-0.7pt}}, preaction={decorate},postaction={draw, line width=2pt, white, shorten >= 1.5mm}]
	
	\tikzstyle{NLTO}  		= [semithick, double distance=2pt, shorten <=0.9pt, shorten >=2mm, decoration={markings,mark=at position 1 with {\arrow[semithick]{open triangle 60}}}, preaction={decorate}, postaction={draw, line width=2pt, white, shorten >= 1.5mm}]
	
	\tikzstyle{NLTONL}  		= [semithick, double distance=2pt, shorten <=0.9pt, shorten >=2.2mm, decoration={markings,mark=at position 1 with {\arrow[semithick]{open triangle 60}}}, preaction={decorate},postaction={draw, line width=2pt, white, shorten >= 1.5mm}]
	
	\tikzstyle{innerWhite} = [semithick, white,line width=2pt, shorten >= 2mm, shorten <= 2mm]
	
	%\tikzstyle{TO}  		= [semithick, double distance=2pt, shorten >=2mm, decoration={markings,mark=at position 1 with {\arrow[semithick]{open triangle 60}}}, postaction={decorate}]
		%
	%\tikzstyle{TONL} = [semithick, double distance=2pt, shorten >=2.2mm, decoration={markings,mark=at position 1 with {\arrow[semithick]{open triangle 60}}, transform={xshift=-0.7pt}}, postaction={decorate}]
	%
	%\tikzstyle{NLTO} = [semithick, double distance=2pt, shorten <=0.79pt, shorten >=2mm, decoration={markings,mark=at position 1 with {\arrow[semithick]{open triangle 60}}}, postaction={decorate}]
	%
	%\tikzstyle{NLTONL} = [semithick, double distance=2pt, shorten <=0.79pt, shorten >=2.2mm, decoration={markings,mark=at position 1 with {\arrow[semithick]{open triangle 60}}, transform={xshift=-0.7pt}}, postaction={decorate}]
	

	% Linien ohne Pfeil 
	\tikzstyle{line}       = [thick]
	\tikzstyle{lineNL}     = [thick, shorten >= 0.6pt]
	\tikzstyle{NLline}     = [thick, shorten <= 0.6pt]
	\tikzstyle{NLlineNL}	 = [thick, shorten <= 0.6pt, shorten >= 0.6pt]
	
	% Doppellinien ohne Pfeil
	\tikzstyle{LINE}       = [semithick, double distance=2pt]
	\tikzstyle{LINENL}     = [semithick, double distance=2pt, shorten >= 0.9pt]
	\tikzstyle{NLLINE}     = [semithick, double distance=2pt, shorten <= 0.9pt]
	\tikzstyle{NLLINENL}	 = [semithick, double distance=2pt, shorten <= 0.9pt, shorten >= 0.9pt]
	
	\tikzstyle{neg}				 = [postaction={decorate,decoration={markings, mark=at position 1 with{\draw[-](-2pt,-3pt)--(-2pt,-7pt);}}}]
	
	\tikzstyle{labelabove} = [above, anchor=base, yshift=0.75ex]
	\tikzstyle{labelbelow} = [below, anchor=base, yshift=-2ex]
	\tikzstyle{labelright} = [right]
	\tikzstyle{labelleft}  = [left]
	
	
	% Pins und Labels
	\tikzstyle{every pin edge}	= [<-, thick]
	\tikzstyle{every pin}	 			= [pin distance=5\mm]
	\tikzstyle{every label}			= [font=\small]
	\tikzstyle{terminal}				= [coordinate]
	
	% extra Symbole f�r circuits-Library
	\tikzstyle{jack}	= [draw, circle, minimum size=1.5mm, inner sep=0]


\newcommand{\BSBnormal}[1]{
	% TikZstyles f�r Blockschaltbilder

\renewcommand{\TikZscale}{1}
\tikzset{every picture/.style={node distance=4\mm, >=stealth'}}

%% Bl�cke
	% Rechteckige Bl�cke
	\tikzstyle{block}      = [draw, semithick, rectangle, minimum height=8\mm, minimum width=8\mm, inner sep=3pt]
	\tikzstyle{NLblock}    = [draw, semithick, rectangle, minimum height=8\mm, minimum width=8\mm, inner sep=3pt, double distance=1.2pt]
	\tikzstyle{PICblock}   = [draw, semithick, rectangle, minimum height=8\mm, minimum width=8\mm, inner sep=2pt]
	\tikzstyle{NLPICblock} = [draw, semithick, rectangle, minimum height=8\mm, minimum width=8\mm, inner sep=3pt, double distance=1.2pt]
	\tikzstyle{noblock}	   = [rectangle, inner sep=-0.6pt]

	% Dreieckige Bl�cke
	\tikzstyle{Rgain}			 = [draw, semithick, isosceles triangle, inner sep=1pt, minimum height=8\mm, isosceles triangle apex angle=60]
	\tikzstyle{Lgain}			 = [draw, semithick, isosceles triangle, inner sep=1pt, minimum height=8\mm, isosceles triangle apex angle=60, shape border rotate=180]
	\tikzstyle{Ugain}			 = [draw, semithick, isosceles triangle, inner sep=1pt, minimum height=8\mm, isosceles triangle apex angle=60, shape border rotate=90]
	\tikzstyle{Dgain}			 = [draw, semithick, isosceles triangle, inner sep=1pt, minimum height=8\mm, isosceles triangle apex angle=60, shape border rotate=-90]

	% Runde Bl�cke
	\tikzstyle{sum}   	   = [draw, semithick, circle, inner sep=1pt, minimum size=3\mm]
	\tikzstyle{branch}		 = [draw, circle, inner sep=0pt, minimum size=1\mm, fill=black]
	\tikzstyle{BRANCH}		 = [coordinate]


%% Verbindungselemente
	% Linien mit Pfeil
	\tikzstyle{to}  		= [->, thick]
	\tikzstyle{toNL}		= [->, thick, shorten >=0.9pt]
	\tikzstyle{NLto}		= [->, thick, shorten <=0.9pt]
	\tikzstyle{NLtoNL}	= [->, thick, shorten <=0.9pt, shorten >=0.9pt]
	
	\tikzstyle{TO}  		= [semithick, double distance=2pt, shorten >=2mm, decoration={markings,mark=at position 1 with {\arrow[semithick]{open triangle 60}}}, preaction={decorate},postaction={draw, line width=2pt, white, shorten >= 1.5mm}]
	
	\tikzstyle{TONL}  		= [semithick, double distance=2pt, shorten >=2.2mm, decoration={markings,mark=at position 1 with {\arrow[semithick]{open triangle 60}}, transform={xshift=-0.7pt}}, preaction={decorate},postaction={draw, line width=2pt, white, shorten >= 1.5mm}]
	
	\tikzstyle{NLTO}  		= [semithick, double distance=2pt, shorten <=0.9pt, shorten >=2mm, decoration={markings,mark=at position 1 with {\arrow[semithick]{open triangle 60}}}, preaction={decorate}, postaction={draw, line width=2pt, white, shorten >= 1.5mm}]
	
	\tikzstyle{NLTONL}  		= [semithick, double distance=2pt, shorten <=0.9pt, shorten >=2.2mm, decoration={markings,mark=at position 1 with {\arrow[semithick]{open triangle 60}}}, preaction={decorate},postaction={draw, line width=2pt, white, shorten >= 1.5mm}]
	
	\tikzstyle{innerWhite} = [semithick, white,line width=2pt, shorten >= 2mm, shorten <= 2mm]
	
	%\tikzstyle{TO}  		= [semithick, double distance=2pt, shorten >=2mm, decoration={markings,mark=at position 1 with {\arrow[semithick]{open triangle 60}}}, postaction={decorate}]
		%
	%\tikzstyle{TONL} = [semithick, double distance=2pt, shorten >=2.2mm, decoration={markings,mark=at position 1 with {\arrow[semithick]{open triangle 60}}, transform={xshift=-0.7pt}}, postaction={decorate}]
	%
	%\tikzstyle{NLTO} = [semithick, double distance=2pt, shorten <=0.79pt, shorten >=2mm, decoration={markings,mark=at position 1 with {\arrow[semithick]{open triangle 60}}}, postaction={decorate}]
	%
	%\tikzstyle{NLTONL} = [semithick, double distance=2pt, shorten <=0.79pt, shorten >=2.2mm, decoration={markings,mark=at position 1 with {\arrow[semithick]{open triangle 60}}, transform={xshift=-0.7pt}}, postaction={decorate}]
	

	% Linien ohne Pfeil 
	\tikzstyle{line}       = [thick]
	\tikzstyle{lineNL}     = [thick, shorten >= 0.6pt]
	\tikzstyle{NLline}     = [thick, shorten <= 0.6pt]
	\tikzstyle{NLlineNL}	 = [thick, shorten <= 0.6pt, shorten >= 0.6pt]
	
	% Doppellinien ohne Pfeil
	\tikzstyle{LINE}       = [semithick, double distance=2pt]
	\tikzstyle{LINENL}     = [semithick, double distance=2pt, shorten >= 0.9pt]
	\tikzstyle{NLLINE}     = [semithick, double distance=2pt, shorten <= 0.9pt]
	\tikzstyle{NLLINENL}	 = [semithick, double distance=2pt, shorten <= 0.9pt, shorten >= 0.9pt]
	
	\tikzstyle{neg}				 = [postaction={decorate,decoration={markings, mark=at position 1 with{\draw[-](-2pt,-3pt)--(-2pt,-7pt);}}}]
	
	\tikzstyle{labelabove} = [above, anchor=base, yshift=0.75ex]
	\tikzstyle{labelbelow} = [below, anchor=base, yshift=-2ex]
	\tikzstyle{labelright} = [right]
	\tikzstyle{labelleft}  = [left]
	
	
	% Pins und Labels
	\tikzstyle{every pin edge}	= [<-, thick]
	\tikzstyle{every pin}	 			= [pin distance=5\mm]
	\tikzstyle{every label}			= [font=\small]
	\tikzstyle{terminal}				= [coordinate]
	
	% extra Symbole f�r circuits-Library
	\tikzstyle{jack}	= [draw, circle, minimum size=1.5mm, inner sep=0]

	\input{#1}
}