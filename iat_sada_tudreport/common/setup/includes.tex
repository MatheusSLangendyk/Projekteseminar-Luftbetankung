% Dieses File dient zum Einbinden wichtiger und nützlicher Pakete.
% Nicht alle Pakete müssen verwendet werden.
%
\usepackage[T1]{fontenc}	% Für Silbentrennung bei Wörten mit Sonderzeichen (z.B. Umlaute)
\usepackage[utf8]{inputenc}	% Um Sonderzeichen (ä, ß, é, ...) direkt eingeben zu können
\usepackage[english,main=ngerman]{babel}
							% Für Sprachenspezifisches
							% ngerman ist schon als globale Option definiert
\usepackage{ifclass}
\usepackage{iflang}			% für sprachabhängige Einstellungen
\usepackage{xkvltxp}		% für Macros in key-value Paketoptionen

%\usepackage{helvet}		% Helvetica als Standard-Sans-Schriftart
\usepackage[stable]{footmisc}
\usepackage{booktabs}


\usepackage{graphicx}		% zum Einbinden von Postscript
\usepackage{psfrag}			% Beschriftung der Bilder
\ifbeamer
	\usepackage{amsmath}% Mehr mathematischer Formelsatz
\else
	\usepackage[tbtags]{amsmath}% Mehr mathematischer Formelsatz
\fi
%\usepackage{amssymb}		% Mehr mathematische Symbole
%\usepackage{amsthm}

\usepackage{float}			% Für Parameter [H] bei Fließobjekten

\usepackage{epsfig}			% Um eps-Bilder einzubinden
\usepackage{scrhack}		% Um Warnung "float@addtolists detected" zu unterdrücken 
\usepackage{subcaption}		% Für Unterabbildungen
\usepackage{ltxtable} 		% Vereinigt TabularX und Longtable
\usepackage{upgreek}		% Für nicht-kursive kleine griechischen Buchstaben
\usepackage{uniinput}		% Diverse Makros zum Ersetzen von UTF8 Zeichen durch Latex äquivalente					
\usepackage{multirow}		% Für mehrzeilige Felder in Tabellen
\usepackage{rotating}		% Zum Drehen von Objekten
\usepackage{icomma}			% Damit nach Dezimalkommas kein Abstand eingefügt wird
\usepackage{dcolumn}		% Zum Ausrichten von Tabellenspalten am Dezimaltrennzeichen

%\usepackage{wrapfig}		% Für kleine Bilder am Rand
%\usepackage{floatflt}		% Alternative zu wrapfig
%\usepackage[hang]{caption}	% Damit mehrzeilige Bildunterschriften gut aussehen
\usepackage{textcomp}		% Für Sonderzeichen im normalen Text
							% (offensichtlich in tudreport schon eingebunden)
\usepackage[ngerman]{varioref}		% Für vref
\usepackage{color}			% Für farbigen Text
\usepackage{placeins}		% Für \FloatBarrier
\usepackage{xspace}
\usepackage{cancel}			% Zum Wegstreichen von Gleichungstermen \cancel{a} oder \cancelto{0}{a}
\usepackage{listings}		% Um formatierten Quellcode einzubinden


\usepackage{array}			% Für Zellentyp "m{}" in tabular-Umgebungen (Vertikal zentriert)
\usepackage{moreverb}		% Für Umgebung "`verbatimtab"' (Verbatim mit Tabs)
\renewcommand{\verbatimtabsize}{4\relax}	% Standardtabweite in "`verbatimtab"'
											% ist 4 Zeichen

\usepackage{abk}

\usepackage{siunitx}		% Um Zahlen mit Einheiten korrekt darzustellen \SI{12}{\meter\per\second} 
\sisetup{
	range-units=single,
	range-phrase={\,\textnormal{\textendash}\,},
	output-decimal-marker={,},
	%per-mode=fraction,
	per-mode=reciprocal,
	per-mode=symbol,
	math-micro=\muup,% Im TU-Design passt das µ sonst nicht
	math-ohm  =\Omegaup,
	text-micro={\fontfamily{mdbch}\textmu},
	text-ohm  ={\fontfamily{mdbch}\textohm}
}

% ======== Glossar =========
\newif\ifSADAuseglossaries
\SADAuseglossariestrue
\SADAuseglossariesfalse
\ifSADAuseglossaries
	\usepackage[automake, acronym, symbols, toc, translate=babel]{glossaries}
	\addto\captionsngerman{%
		\renewcommand*{\acronymname}{Abkürzungsverzeichnis}
		\renewcommand*{\glssymbolsgroupname}{Symbolverzeichnis}
	}
\fi

% ======== Anfang Literaturverwaltung =====
%\usepackage[datename]{babelbib}	% Für deutsche Literaturverwaltung
%\setbtxfallbacklanguage{ngerman}
%\btxprintISSN{false}
%\setbibliographyfont{lastname}{\textsc}%
\usepackage[babel]{csquotes}
\ifbeamer
	\usepackage[
		backend=biber,
		style=iatsada-numeric,
		backref=false
	]{biblatex}					% Für deutsche Literaturverwaltung
\else
	\usepackage[
		backend=biber,
		style=iatsada-numeric,
		backref=true
	]{biblatex}					% Für deutsche Literaturverwaltung
	% ==========================================
\fi

\ifbeamer
	% Hier Pakete die nur für Präsentationen benötigt werden
	\usepackage{animate}
	\usepackage{multimedia}
\fi

\ifbeamer
	\usepackage{todonotes}		% Für Anmerkungen \todo{Bitte ändern} bzw. \todo[inline]{Bitte ändern}
\else
	\usepackage[textwidth=1.8cm, textsize=tiny]{todonotes}		% Für Anmerkungen \todo{Bitte ändern} bzw. \todo[inline]{Bitte ändern}	
	%muss eigentlich hinter begin{document} stehen, macht dann aber die Fusszeile kaputt
	%\setlength{\marginparwidth}{1.8cm} % notwendig damit todos am Rand korrekt dargestellt werden (siehe Doku todonotes Paket)
\fi

% Das Packet hyperref immer als letztes einbinden (nur bookmark darf danach kommen)!
% Für Verlinkungen im erzeugten pdf
\ifbeamer
	\usepackage[colorlinks=false]{hyperref}
\else
	\usepackage[unicode=true, colorlinks=false, breaklinks=true]{hyperref}
	\ifSADATUbama
		\usepackage[a-1b]{pdfx}
	\fi
\fi
\usepackage{bookmark}