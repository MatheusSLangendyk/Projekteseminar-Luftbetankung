% Diese Datei dient zum definieren nützlicher Befehle.
% Sie soll lediglich als Beispiel dienen, wie Befehle definiert werden, und welche Befehle nützlich sein können.
\newcommand{\eqp}{\ensuremath{\, \, \, .}}
\newcommand{\M}[1]{\textbf{#1}} %Matrix  M
\newcommand{\Mr}[1]{\textbf{#1}_\rho} %Matrix mit tiefgestelltem rho, M_rho
\newcommand{\Mtil}[1]{\tilde{\textbf{#1}}} %Matrix mit Tilde
\newcommand{\Mtilt}[2]{\tilde{\textbf{#1}}_{#2}} %Matrix mit Tilde mit tiefgestelltem arg2
\DeclareRobustCommand{\Mt}[2]{\textbf{#1}_{#2}} %Matrix M mit tiefgestelltem arg2


\DeclareRobustCommand{\w}[1]{\underline{#1}} %Vektor arg1 unterstrichen 
\DeclareRobustCommand{\wt}[2]{\underline{#1}_{#2}} %Vektor unterstrichen w mit tiefgestelltem arg2
\DeclareRobustCommand{\wr}[1]{\underline{#1}_{\rho}} %Vektor unterstrichen mit rho, w_rho

% Inhalt
% ======
%	Makros für Referenzen (Abbildungen, Zitate, ...)
%	Makros für Abbildungen
%	Makros für Einheiten, Exponenten
%	Makros für Formeln
%	Makros für Entwurf
%   Definitionen für Umgebungen

% Makros für Abbildungen
% ======================
	% zum Skalieren nach Ersetzen durch psfrag
	\newcommand*{\incgraphicsw}[2]{\resizebox{#1}{!}{\includegraphics{#2}}}


% Textbausteine
% =============
	% Produktnamen
	\newcommand*{\Matlab}{\textsc{Matlab}}
	\newcommand*{\Matlabreg}{\textsc{Matlab}\textsuperscript{\tiny \textregistered}}
	\newcommand*{\MatSim}{\textsc{Matlab/Simulink}}
	\newcommand*{\Simulink}{\textsc{Simulink}}
	\newcommand*{\Simulinkreg}{\textsc{Simulink}\textsuperscript{\tiny \textregistered}}
	
	% Das Makro |\name|\marg{person} formatiert einen Personennamen bspw. eines Erfinders oder Entdeckers gemäß |\name{Euler}| \arrow\ \name{Euler}.
	\newcommand*{\name}[1]{\textsc{#1}}



% Makros für Einheiten, Exponenten
% ================================

	\newcommand*{\unit}[1]{\ensuremath{\mathrm{#1}}}
	
	% Wert mit Einheit (mit kleinem Leerzeichen dazwischen), aus Text- UND Math-Modus
	\newcommand*{\valunit}[2]{\ensuremath{#1\,\mrm{#2}}}


	% "°C", im Text- oder Mathe-Modus
	\newcommand*{\degC}{
		\ifmmode
			^\circ \mrm{C}%
		\else
			\textdegree C%
		\fi}

	\newcommand*{\degree}{
		\ifmmode
			^\circ%
		\else
			\textdegree%
		\fi}
	
	% Für Exponentenschreibweise ( Anwendung: 123\E{3} )
	\newcommand*{\E}[1]{\ensuremath{\cdot 10^{#1}}}
	
	\newcommand*{\eexp}[1]{\ensuremath{\mathrm{e}^{#1}}}
	\newcommand*{\iu}{\ensuremath{\mathrm{j}}}

	\newcommand*{\todots}{\ensuremath{,\,\hdots,\,}}


% Makros für Formeln
% ==================

	% Definition für Vektor und Matizen
    \newcommand*{\mat}[1]{{\ensuremath{\boldsymbol{\mathrm{#1}}}}}
    \newcommand*{\ma}[1]{{\ensuremath{\boldsymbol{\mathrm{#1}}}}}
    \newcommand*{\mas}[1]{\ensuremath{\boldsymbol{#1}}}
    \newcommand*{\ve}[1]{\ensuremath{\boldsymbol{#1}}}
    \newcommand*{\ves}[1]{\ensuremath{\boldsymbol{\mathrm{#1}}}}

	\newcommand*{\AP}{\ensuremath{\mathrm{AP}}}
	\newcommand*{\doti}{\ensuremath{(i)^\cdot}}
	
	\newcommand*{\inprod}[2]{\ensuremath{\langle #1,\,#2 \rangle}}
	
	\newcommand*{\ul}[1]{\underline{#1}}

	% gerades "d" (z.B. für Integral)
	\newcommand*{\ud}{\ensuremath{\mathrm{d}}}
	
	% normaler Text in Formeln
	\newcommand*{\tn}[1]{\textnormal{#1}}
	
	% nicht-kursive Schrift in Formeln
	\newcommand*{\mrm}[1]{\ensuremath{\mathrm{#1}}}
	
	% gerades "T" für Transponiert
	\newcommand*{\transp}{\ensuremath{\mathrm{T}}}
	
	% gerades "rg"
	\newcommand*{\rang}{\ensuremath{\operatorname{rg}}}

	% Für geklammerte Ausdrücke mit Index (Subscript)
	% (einmal mit kursiven Index, einmal mit geradem Index)
	\newcommand*{\grpsb}[2]{\ensuremath{\left(#1\right)_{#2}}}
	\newcommand*{\grprsb}[2]{\ensuremath{\left(#1\right)_{\mathrm{#2}}}}

	% Ableitungen und Integrale
		% "normale" Ableitung (mit geraden "d"s)
		\newcommand*{\normd}[2]{\ensuremath{\frac{\mathrm{d}#1}{\mathrm{d}#2}}}
		\newcommand*{\normdat}[3]{\ensuremath{\left.\frac{\mathrm{d} #1}{\mathrm{d} #2}\right|_{#3}}}
	
		% Materielle Ableitung
		\newcommand*{\matd}[2]{\ensuremath{\frac{\mathrm{D} #1}{\mathrm{D} #2}}}
		\newcommand*{\matdat}[3]{\ensuremath{\left.\frac{\mathrm{D} #1}{\mathrm{D} #2}\right|_{#3}}}
	
		% Partielle Ableitung
		\newcommand*{\partiald}[2]{\ensuremath{\frac{\partial #1}{\partial #2}}}
		\newcommand*{\partialdat}[3]{\ensuremath{\left.\frac{\partial #1}{\partial #2}\right|_{#3}}}
	
	
	% Transformationen
	\newcommand*{\FT}[1]{\ensuremath{\mathfrak{F}\left\{#1\right\}}}
	\newcommand*{\FTabs}[1]{\ensuremath{\left|\mathfrak{F}\left\{#1\right\}\right|}}
	\newcommand*{\IFT}[1]{\ensuremath{\mathfrak{F}^{-1}\left\{#1\right\}}}
	\newcommand*{\DFT}[1]{\ensuremath{\mathrm{DFT}\left\{#1\right\}}}
	\newcommand*{\DFTabs}[1]{\ensuremath{\left|\mathrm{DFT}\left\{#1\right\}\right|}}
	\newcommand*{\Laplace}[1]{\ensuremath{\mathfrak{L}\left(#1\right)}}
	\newcommand*{\InvLaplace}[1]{\ensuremath{\mathfrak{L^{-1}}\left(#1\right)}}
	\newcommand*{\invtrans}{\ensuremath{\quad\bullet\!\!-\!\!\!-\!\!\circ\quad}}
	\newcommand*{\trans}{\ensuremath{\quad\circ\!\!-\!\!\!-\!\!\bullet\quad}}


	\newcommand*{\mlfct}[1]{\texttt{#1}}
	\newcommand*{\mlvar}[1]{\texttt{#1}}


	% Manche textcomp-Zeichen funktionieren mit dem TU-Design nicht, diese können dann mit diesem
	% Befehl gesetzt werden.
	\newcommand*{\textcompstdfont}[1]{{\fontfamily{cmr} \fontseries{m} \fontshape{n} \selectfont #1}}
	


% =================================================================================
% Defintionen für Mathe-Umgebungen
% =================================================================================
	
\ifcsmacro{theorem}{}{
	\newtheorem{theorem}{Satz}
}
\ifcsmacro{lemma}{}{
	\newtheorem{lemma}[theorem]{Lemma}	% Selber Zähler wie theorem
}
\ifcsmacro{definition}{}{
	\newtheorem{definition}{Definition}
}
% =================================================================================


% =================================================================================
% Defintionen für Beispiel-Umgebung
% =================================================================================
\makeatletter
\ifx\c@chapter\undefined
	\newcounter{chapter}
\fi
\ifx\c@examplenumber\undefined
	\newcounter{examplenumber}[chapter]% Neuer Counter bspnummer nummeriert nach Kapitel
\fi
\makeatother
\def\theexamplenumber{\thechapter.\arabic{examplenumber}}

\ifcsmacro{example}{}{
	\newenvironment{example}[1][]
	{\vskip 3\parskip plus 1pt minus 1pt \refstepcounter{examplenumber}
	\vspace{.3cm} \begin{addmargin}[1cm]{0cm} \noindent \textbf{Beispiel \theexamplenumber}: \textbf{#1}\par}
	{\end{addmargin} \par \vspace{.3cm}}
}

% Alternative, einfachere Beispielumgebung:
% \newtheorem{example}{Beispiel}
% =================================================================================

% =================================================================================
% Definitionen für Tabellen
% =================================================================================
% Spaltenstil zum Ausrichten von Zahlen am Dezimaltrennzeichen
\newcolumntype{d}[1]{D{.}{,}{#1}}


% =================================================================================
% Definitionen für Listingsumgebung
% =================================================================================

\lstloadlanguages{Matlab}

\lstdefinestyle{Matlab_colored_smallfont}{
	language = Matlab,
	keywords = {break,case,catch,continue,class,else,elseif,end,enumeration,for,function,global,if,methods,otherwise,persistent,properties,return,switch,try,while},
	tabsize = 4,
	framesep = 3mm,
	frame=tb,
	classoffset = 0,	
	basicstyle = \footnotesize\ttfamily,
	keywordstyle = \bfseries\color[rgb]{0,0,1},
	commentstyle = \itshape\color[rgb]{0.133,0.545,0.133},
	stringstyle = \color[rgb]{0.627,0.126,0.941},
	extendedchars = true,
	breaklines = true,
	prebreak = \textrightarrow,
	postbreak = \textleftarrow,
	%escapeinside = {(*@}{@*)},
	%moredelim = [s][\itshape\color[rgb]{0.5,0.5,0.5}]{[.}{.]},
	numbers = left,
	numberstyle = \tiny,
	stepnumber = 5
}

\lstdefinestyle{Matlab_colored}{
	language = Matlab,
	keywords = {break,case,catch,continue,class,else,elseif,end,enumeration,for,function,global,if,methods,otherwise,persistent,properties,return,switch,try,while},
	tabsize = 4,
	framesep = 3mm,
	frame=tb,
	classoffset = 0,	
	basicstyle = \ttfamily,
	keywordstyle = \bfseries\color[rgb]{0,0,1},
	commentstyle = \itshape\color[rgb]{0.133,0.545,0.133},
	stringstyle = \color[rgb]{0.627,0.126,0.941},
	extendedchars = true,
	breaklines = true,
	prebreak = \textrightarrow,
	postbreak = \textleftarrow,
	%escapeinside = {(*@}{@*)},
	%moredelim = [s][\itshape\color[rgb]{0.5,0.5,0.5}]{[.}{.]},
	numbers = left,
	numberstyle = \tiny,
	stepnumber = 5
}


\lstdefinestyle{C_colored_smallfont}{
	language=C,
	tabsize = 4,
	framesep = 3mm,
	frame=tb,	
	classoffset = 0,	
	basicstyle = \footnotesize\ttfamily,
	keywordstyle = \bfseries\color[rgb]{0,0,1},
	commentstyle = \itshape\color[rgb]{0.133,0.545,0.133},
	stringstyle = \color[rgb]{0.627,0.126,0.941},
	extendedchars = true,
	breaklines = true,
	prebreak = \textrightarrow,
	postbreak = \textleftarrow,
	%escapeinside = {(*@}{@*)},
	%moredelim = [s][\itshape\color[rgb]{0.5,0.5,0.5}]{[.}{.]},
	numbers = left,
	numberstyle = \tiny,
	stepnumber = 5
}

\lstdefinestyle{C_colored}{
	language=C,
	tabsize = 4,
	framesep = 3mm,
	frame=tb,
	classoffset = 0,	
	basicstyle = \ttfamily,
	keywordstyle = \bfseries\color[rgb]{0,0,1},
	commentstyle = \itshape\color[rgb]{0.133,0.545,0.133},
	stringstyle = \color[rgb]{0.627,0.126,0.941},
	extendedchars = true,
	breaklines = true,
	prebreak = \textrightarrow,
	postbreak = \textleftarrow,
	%escapeinside = {(*@}{@*)},
	%moredelim = [s][\itshape\color[rgb]{0.5,0.5,0.5}]{[.}{.]},
	numbers = left,
	numberstyle = \tiny,
	stepnumber = 5
}

% Unterstützung von Umlauten in listings und tcblistings (tcolorbox Paket)
\lstset{
	inputencoding=utf8,
	extendedchars=true,
	literate=%
		{á}{{\'a}}1
		{é}{{\'e}}1
		{í}{{\'i}}1
		{ó}{{\'o}}1
		{ú}{{\'u}}1
		{Á}{{\'A}}1
		{É}{{\'E}}1
		{Í}{{\'I}}1
		{Ó}{{\'O}}1
		{Ú}{{\'U}}1
		{à}{{\`a}}1
		{è}{{\`e}}1
		{ì}{{\`i}}1
		{ò}{{\`o}}1
		{ù}{{\`u}}1
		{À}{{\`A}}1
		{È}{{\'E}}1
		{Ì}{{\`I}}1
		{Ò}{{\`O}}1
		{Ù}{{\`U}}1
		{ä}{{\"a}}1
		{ë}{{\"e}}1
		{ï}{{\"i}}1
		{ö}{{\"o}}1
		{ü}{{\"u}}1
		{Ä}{{\"A}}1
		{Ë}{{\"E}}1
		{Ï}{{\"I}}1
		{Ö}{{\"O}}1
		{Ü}{{\"U}}1
		{â}{{\^a}}1
		{ê}{{\^e}}1
		{î}{{\^i}}1
		{ô}{{\^o}}1
		{û}{{\^u}}1
		{Â}{{\^A}}1
		{Ê}{{\^E}}1
		{Î}{{\^I}}1
		{Ô}{{\^O}}1
		{Û}{{\^U}}1
		{œ}{{\oe}}1
		{Œ}{{\OE}}1
		{æ}{{\ae}}1
		{Æ}{{\AE}}1
		{ß}{{\ss}}2
		{ç}{{\c c}}1
		{Ç}{{\c C}}1
		{ø}{{\o}}1
		{å}{{\r a}}1
		{Å}{{\r A}}1
		{€}{{\EUR}}1
		{£}{{\pounds}}1
}

