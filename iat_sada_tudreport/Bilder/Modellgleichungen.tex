\documentclass[10pt,a4paper]{article}

\usepackage{ngerman}
\usepackage[ngerman]{hyperref}
\usepackage{caption}
\usepackage{mathtools}
\usepackage{mathptmx}
\usepackage{subcaption}
\usepackage{graphicx}
\usepackage{placeins}
\usepackage{amsmath}
\usepackage{apacite}
\usepackage[stable]{footmisc}
\begin{document}
\section{Modellgleichungen}

\begin{center}
 \begin{tabular}{||c c c c ||} 
 \hline
 Zustand & Symbol & Einheit & Koordinatensystem \\ [0.5ex] 
 \hline\hline
 Körpergeschwindigkeit $x$-Richtung& $u$ & $[m/s]$& Körperfest\\ 
 \hline
 Körpergeschwindigkeit $y$-Richtung& $v$ & $[m/s]$& Körperfest\\ 
 \hline
 Körpergeschwindigkeit $z$-Richtung& $w$ & $[m/s]$& Körperfest\\ 
 \hline
 Körperdrehgeschwindigkeit  $x$-Achse& $p$ & $[rad/s]$& Körperfest\\ 
 \hline
 Körperdrehgeschwindigkeit $y$-Achse& $q$ & $[rad/s]$& Körperfest\\ 
 \hline
 Körperdrehgeschwindigkeit  $z$-Achse& $r$ & $[rad/s]$& Körperfest\\ 
 \hline
 Rollwinkel & $\phi$ & $[rad]$& Körperfest\\ 
 \hline
 Nickwinkel & $\theta$ & $[rad]$& Körperfest\\ 
 \hline
 Gierwinkel & $\psi$ & $[rad]$& Körperfest\\ 
 \hline
 Anstellwinkel & $\alpha$ & $[rad]$& Körperfest\\ 
 \hline
 Schiebewinkel & $\beta$ & $[rad]$& Körperfest\\ 
 \hline
 Position Nord & $x$ & $[m]$& Erdkoordinaten (Flache Erde)\\ 
 \hline
 Position Ost & $y$ & $[m]$& Erdkoordinaten (Flache Erde)\\ 
 \hline
 Position Unten & $z$ & $[m]$& Erdkoordinaten (Flache Erde)\\ 
 \hline
 Höhenruderwinkel & $\eta$ & $[rad]$& Körperfest\\ 
 \hline
 Normierte Schubkraft  & $\delta$ & -& -\\ 
 \hline
 Querruderwinkel & $\xi$ & $[rad]$& Körperfest\\ 
 \hline
 Seitenruderwinkel & $\zeta$ & $[rad]$& Körperfest\\ [1ex] 
 \hline
\end{tabular}
\end{center}
\subsection{Zustände und Stellgröße}
\begin{itemize}
\item Körpergeschwindigkeit $\underline{v} = [u, v, w]^T$
\item Körperdrehgeschwindigkeit $\underline{\omega} = [p, q, r]^T$
\item Eulerwinkel (ohne den Gierwinkel) $\underline{\varphi} = [\phi, \theta]^T$
\item Höhe $h = -z$.

\end{itemize}
Der gesamte Zustand lässt sich als 
\begin{equation}
\underline{x} = \begin{bmatrix} 
\underline{v} \\
\underline{\omega} \\
\underline{\varphi}\\
h
\end{bmatrix} = \begin{bmatrix} 
u\\v\\w\\p\\q\\r\\ \phi\\ \theta\\ h
\end{bmatrix},
\end{equation}
darstellen.\\
Die Stellgrößen werden als 
\begin{equation}
\underline{u} = \begin{bmatrix} 
\eta \\ \delta \\ \xi \\ \zeta
\end{bmatrix},
\end{equation} 
definiert, wo die Schubkraft anhand der Gewichtskraft des Flugzeugs normiert wird: $\delta = \dfrac{F_\mathrm{schub}}{mg}$
\subsection{Sekundäre Gleichungen}
\label{sec:Anströmung}
\begin{itemize}
\item Fluggeschwindigkeitsvektor $\underline{v}_\mathrm{A} = \underline{v}- \textbf{T}_\mathrm{fg}\underline{v}^\mathcal{G}_\mathrm{w}  = [u_\mathrm{A},v_\mathrm{A},w_\mathrm{A}]^T $, wo $\underline{v}^\mathcal{G}_\mathrm{w}$ der Windgeschwindigkeitsvektor in geodätischem Koordinatensystem und $\textbf{T}_\mathrm{fg}$ die Körperfestetransformationsmatrix sind. Die Windgeschwindigkeit wird ,im Rahmen des Projekts, als externe Störung betrachtet (Kapitel \ref{chap:ZweiFlieger}) und wird daher nicht als Teil des Modells weiter berücksichtigt.  
\item Betrag der Anströmungsgeschwindigkeit $V_\mathrm{A} = |\underline{v}_\mathrm{A}| $.
\item Anstellwinkel $\alpha = \tan^{-1}(\dfrac{w_\mathrm{A}}{u_\mathrm{A}})$
\item Schiebewinkel $\beta = \sin^{-1}(\dfrac{v_\mathrm{A}}{V_\mathrm{A}})$
\item Staudruck $q = \dfrac{1}{2}\rho V_\mathrm{A}^2$

\end{itemize}
\subsection{Aerodynamische Beiwerte}
obs: Alle nicht-definierte Parameter sind Geometriekonstanten.\\

\subsubsection{Translatorische Beiwerte}
\textbf{Antriebsbeiwert}\\

\begin{itemize}
\item Antrieb an der Flügelfläche: $C_\mathrm{A_F}^\mathcal{A} = K_\alpha\alpha + \alpha_0$
\item Abwind an dem Höhenruder $\epsilon = K_{\epsilon}(\alpha-\alpha_\mathrm{0})$
\item Anstellwinkel an dem Höhenruder $\alpha_\mathrm{h} = \alpha - \epsilon-\eta + q\dfrac{l_\mathrm{t}}{V_\mathrm{A}}$
\item Antrieb an dem Höhenruder: $C_\mathrm{A_H}^\mathcal{A} = 3.1 \alpha_\mathrm{h}\dfrac{S_\mathrm{H}}{S}$
\item Gesamtauftrieb $C_\mathrm{A}^\mathcal{A} = C_\mathrm{A_F}+C_\mathrm{A_H} $
\end{itemize}
\textbf{Widerstandsbeiwert}\\
$C_\mathrm{W}^\mathcal{A} = 0.13 + \kappa(5.5+0.654)^2$\\
\textbf{Querkraftbeiwert}\\
$C_\mathrm{Q}^\mathcal{A} = -1.6\beta + 0.24\zeta$

\subsubsection{Rotatorische Beiwerte }
\begin{itemize}
\item Vektor des statischen Verhaltens : $\underline{n} = \begin{bmatrix} 
-1.4\beta \\
-0.59 -3.1(\alpha-\epsilon)\dfrac{S_\mathrm{H}l_\mathrm{H}}{Sc} \\
(1 - \alpha\dfrac{180}{15\pi})\beta
\end{bmatrix} $ 
\\
\item Jakobimatrix des dynamischen Drehverhaltens $\dfrac{\partial  \underline{c}_\mathrm{m}}{\partial  \underline{c}_x} = \dfrac{c}{V_\mathrm{A}}\begin{bmatrix} 
-11 & 0 & 5 \\
0 &-4.03\dfrac{S_\mathrm{H}l_\mathrm{H}^2}{Sc}&0 \\
1.7 & 0 & -11.5
\end{bmatrix} $  
\\
\item Jakobimatrix des Steuerverhaltens $\dfrac{\partial  \underline{c}_\mathrm{m}}{\partial  \underline{c}_u} = \begin{bmatrix} 
-0.6 & 0 & 0.22 \\
0 &-3.1\dfrac{S_\mathrm{H}l_\mathrm{H}}{Sc}&0 \\
0 & 0 & -0.63
\end{bmatrix} $  
\item Winkelsteuervektor $\tilde{\underline{u}} = \begin{bmatrix} 
\xi\\
\eta\\
\zeta
\end{bmatrix} $  
\item Moment Beiwertvektor $ \underline{c}_\mathrm{m}  = \begin{bmatrix} 
C_\mathrm{l}\\
C_\mathrm{m}\\
C_\mathrm{n}
\end{bmatrix}=\underline{n} + \dfrac{\partial  \underline{c}_\mathrm{m}}{\partial  \underline{c}_x} \underline{\omega} + \dfrac{\partial  \underline{c}_\mathrm{m}}{\partial  \underline{c}_u} \tilde{\underline{u}}  $
\end{itemize}
\subsection{Kräfte und Momente}
\subsubsection{Triebwerk}
\begin{itemize}
\item $F_\mathrm{res} = \delta mg$
\item $\underline{r}^\mathcal{F}_\mathrm{schub} = \begin{bmatrix} 
F_\mathrm{res}\\
0\\
0
\end{bmatrix}$ 
\end{itemize}
\subsubsection{Aerodynamische Kräfte}
\begin{itemize}
\item Auftriebskraft: $A = qSC_\mathrm{A}$
\item Widerstandskraft: $W = qSC_\mathrm{W}$
\item Querkraft: $Q = qSC_\mathrm{Q}$
\item $\underline{r}^\mathcal{A}_\mathrm{a} = \begin{bmatrix} 
-W\\
Q\\
-A
\end{bmatrix}$  
\item $\underline{r}^\mathcal{F}_\mathrm{a} = \textbf{T}_\mathrm{fa}\underline{r}^\mathcal{A}_\mathrm{a}$
\item Gesamtkraft: $\underline{r}^\mathcal{F}_\mathrm{g} = \underline{r}^\mathcal{F}_\mathrm{a} + \underline{r}^\mathcal{F}_\mathrm{schub}$
\end{itemize}
\subsubsection{Moment Triebwerk}
\begin{itemize}
\item Triebwerkmoment: $\underline{q}^\mathcal{F}_\mathrm{schub} = (\underline{p}_\mathrm{schub}-\underline{p}_\mathrm{sp})\times\underline{r}^\mathcal{F}_\mathrm{schub}$
\end{itemize}
\subsubsection{Aerodynamisches Moment}
\begin{itemize}
\item Rollmoment: $L_A = qSbC_\mathrm{l}$
\item Nickmoment: $M_A = qScC_\mathrm{m}$
\item Giermoment: $N_A = qSbC_\mathrm{n}$
\item Gesamtes aerodynamisches Moment::\\ $\underline{q}^\mathcal{F}_\mathrm{a} = \begin{bmatrix} 
L_\mathrm{A}\\
M_\mathrm{A}\\
N_\mathrm{A}
\end{bmatrix} + (\underline{p}_\mathrm{aero}-\underline{p}_\mathrm{sp})\times\underline{r}^\mathcal{F}_\mathrm{a}$
\item Gesamtmoment $\underline{q}^\mathcal{F}_\mathrm{g} = \underline{q}^\mathcal{F}_\mathrm{a} + \underline{q}^\mathcal{F}_\mathrm{schub} $
\end{itemize}

\subsection{Dynamische Gleichungen}
\begin{itemize}
\item Translatorische Beschleunigung $\underline{\dot{v}} = \dfrac{1}{m}\underline{r}^\mathcal{F}_\mathrm{g} + \textbf{T}_\mathrm{fg}\begin{bmatrix} 
0\\
0\\
g
\end{bmatrix} - \underline{\omega}\times\underline{v}$
\item Rotatorische Beschleunigung $\underline{\dot{\omega}} = -\textbf{I}^{-1}\underline{\omega}\times\textbf{I}\underline{\omega} + \textbf{I}^{-1}\underline{q}^\mathcal{F}_\mathrm{g}$.
\end{itemize}
Da man nur die Winkel $\theta$ und $\phi$ als Teil des Zustands betrachtet, wird eine Transformationsmatrix erforderlich, die die Entfernung des Winkel $\psi$ gewährleistet.
\begin{itemize}
\item Transformationsmatrix $\textbf{T}_\mathrm{winkel} = \begin{bmatrix} 
1 & 0&0\\
0&1&0\\
\end{bmatrix}$
\item Drehrate des geodätischen Koordinatensystems $\textbf{J} = \dfrac{1}{\cos(\theta)}\begin{bmatrix} 
\cos(\theta) & \sin(\theta)\sin(\phi)&\cos(\phi)\sin(\theta)\\
0&\cos(\phi)\cos(\theta)&-\sin(\phi)\cos(\theta)\\
0&\sin(\phi)& \cos(\phi)
\end{bmatrix} $
\item Winkelgeschwindigkeit $\underline{\dot{\varphi}} = \textbf{T}_\mathrm{\phi}\textbf{J}\underline{\omega}$ 
\end{itemize}
Da nur die z-Komponente von dem Positionsvektor gebraucht wird, nutzt man wieder eine Transformationsmatrix, die nur den gewünschten Zustand abliefert.
\begin{itemize}
\item $\textbf{T}_\mathrm{lage} = [1, 0, 0]$
\item Höhenänderung $\dot{h}^\mathcal{G} = -\textbf{T}_\mathrm{lage}\textbf{T}_\mathrm{gf}\underline{v}$
\end{itemize}
Ref:\cite{FlugmechanikBuch}, \cite{RAMPaper}
\bibliographystyle{apacite}
\bibliography{Quellen}
\end{document}\\
